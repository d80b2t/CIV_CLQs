\documentclass[fleqn,usenatbib]{mnras}

\input{format}

\title[High-redshift CLQs]{The first high-redshift Changing Look Quasars}

\author[Ross {\it et al.}]
{Nicholas~P.~Ross$^{1}$\thanks{E-mail: npross@roe.ac.uk},
Matthew Graham$^{2}$,
Giorgio Calderone$^{3}$, 
K. E. Saavik Ford$^{4,5,6}$,  
\newauthor Barry McKernan$^{4,5,6}$  and Daniel Stern$^{7}$  
\\
%%  List of institutions
$^{1}$Institute for Astronomy, University of Edinburgh, Royal Observatory, Blackford Hill, Edinburgh EH9 3HJ, United Kingdom \\
$^{2}$Cahill Center for Astronomy and Astrophysics, California Institute of Technology, Mail Code 249/17, 1200 E California Blvd, Pasadena CA 91125, USA\\
$^{3}$INAF -- Osservatorio Astronomico di Trieste, Via Tiepolo 11, I-34143 Trieste, Italy \\
$^{4}$Department of Science, BMCC, City University of New York, New York, NY 10007, USA \\
$^{5}$Department of Astrophysics, Rose Center for Earth and Space, American Museum of Natural History, Central Park West at 79th Street, NY 10024, USA \\
$^{6}$Graduate Center, City University of New York, 365 5th Avenue, New York, NY 10016, USA\\
$^{7}$Jet Propulsion Laboratory, California Institute of Technology, 4800 Oak Grove Drive, Mail Stop 169-221, Pasadena, CA 91109, USA \\
}

\date{Accepted XXX. Received YYY; in original form ZZZ}

\begin{document}
\label{firstpage}
\pagerange{\pageref{firstpage}--\pageref{lastpage}}
\maketitle

\begin{abstract}
We report on three redshift $z>2$ quasars with dramatic changes in
their \civ\ emission lines, the first ``Changing-Look'' quasars at
high redshift.  This is also the first time the changing-look
behaviour has been seen in a high-ionization emission line.  SDSS
J1205+3422, J1638+2827 and J2228+2201 show interesting behaviour in
their observed optical light curves, and subsequent spectroscopy shows
significant changes in the \civ\ broad emission line, with both line
collapse and emergence being displayed in rest-frame timescales of
$\sim$240-1640 days.  These are very quick changes, especially when
considering virial black hole mass estimates have all three quasars
with $M_{\rm BH} > 10^{9} M_{\odot}$.  Continuum and emission line
measurements from the three quasars show changes in the
continuum-equivalent width plane with the CLQs seen to be on the edge
of the full population distribution, and showing indications of an
intrinsic Baldwin Effect. We put these observations in context with
recent ``state-change'' models, but note that even in their
`low-state', the \civ\ CLQs are above $\sim$10\% in Eddington
luminosity.
\end{abstract}

\begin{keywords}
accretion, accretion discs -- surveys -- quasars: general -- quasars: time-domain: 
\end{keywords}


%%%%%%%%%%%%%%%%%%%%%%%%%%%%%%%%%%%%%%%%%%%%%%%%%%%%%%%%%%%%%%%%%%%%%%%%%%%%%%%%%
%%%%%%%%%%%%%%%%%%%%%%%%%%%%%%%%%%%%%%%%%%%%%%%%%%%%%%%%%%%%%%%%%%%%%%%%%%%%%%%%%
%%
%%
%%   SECTION 1  SECTION 1  SECTION 1  SECTION 1  SECTION 1  SECTION 1  
%%   SECTION 1  SECTION 1  SECTION 1  SECTION 1  SECTION 1  SECTION 1  
%%   SECTION 1  SECTION 1  SECTION 1  SECTION 1  SECTION 1  SECTION 1  
%%
%%
%%%%%%%%%%%%%%%%%%%%%%%%%%%%%%%%%%%%%%%%%%%%%%%%%%%%%%%%%%%%%%%%%%%%%%%%%%%%%%%%%%
%%%%%%%%%%%%%%%%%%%%%%%%%%%%%%%%%%%%%%%%%%%%%%%%%%%%%%%%%%%%%%%%%%%%%%%%%%%%%%%%%%
\section{Introduction}
Luminous AGN, i.e. quasars, are now seen to significantly vary their
energy output on timescales as short as weeks to months.  This
observation, and the subsequent mismatch in the expected ``viscous''
timescale, which for a 10$^{7}$ M$_{\odot}$ central supermassive black
hole (SMBH) is $\sim$hundreds of years, was noted over 30 years ago
\citep[e.g.][]{Alloin1985}. However, with new photometric light-curve
and repeat spectroscopic data, the desire for a deeper understanding
of AGN accretion disk physics has recently re-invigorated the field
\citep[e.g.][]{Antonucci2018, Lawrence2018, Ross2018, Stern2018}.

The optical continuum variability of quasars has been recognized since
their first optical identification
\citep[e.g.,][]{MatthewsSandage1963, MacLeod2012}.  Dramatic changes
in the broad emission lines (BELs) of quasars has only recently been
identified \citep[e.g., ][]{LaMassa2015}.  Samples of over 100
``Changing Look'' quasars (CLQs) or ``Changing State'' quasars (CSQs)
have now been assembled \citep[e.g.][]{MacLeod2019, Graham2019b}. The
community uses both these terms as a cover for the underlying
physics. For sake of argument, CLQs can potentially be thought of as
the extension to the BELs of quasar continuum variability \citep[e.g.,
][]{MacLeod2012} whereas the CSQs have a `state-transition' similar to
that in Galactic X-ray binaries \citep[][]{NodaDone2018, Ruan2019a}. In
this paper, we use the term `Changing Look', as we are currently
agnostic, and confessedly ignorant, to the underlying physical
processes.

\begin{table*}
  \begin{centering}
    \begin{tabular}{l r  r r   lll lll  r r}
      \hline  \hline 
      Line                 & $\lambda$ &  Transition  & Ionization   &  \multicolumn{6}{c}{Transition Levels}                                                                                           & Wavenumber   & $A_{i,j}$                    \\
                              &  / \AA\    & energy / eV &  energy / eV  &    \multicolumn{3}{l}{Lower}    &  \multicolumn{3}{l}{Upper}                                                         & / cm$^{-1}$    & ($\times10^{8}$  / s$^{-1}$)               \\      \hline
      H LyLim           &   912.324   & 13.5984    & 13.5984        & 1$s$                    & $^2$S     & 1/2          & $\infty$                  &                                 &             & 109 678.7       & 1.23$\times10^{-6}$  \\
      H Ly$\alpha$  &  1215.670  & 10.1988    & 13.5984       & 1$s$                     & $^2$S      & 1/2          & 2                             &                                 &             &  82 259.2       &  4.67  \\
      \nv                  &  1238.821  & 10.0082    &  97.8901      &  1$s^{2}$2$s$      &  $^{2}$S   &  1/2         & 1$s^{2}$2$p$          &  $^{2}$P$^{{\rm o}}$ &   3/2    &  80 721.9        & 3.40   \\
      \nv                  &  1242.804  &  9.9762     &  97.8901      &  1$s^{2}$2$s$       &  $^{2}$S   &  1/2        &  1$s^{2}$2$p$         &  $^{2}$P$^{{\rm o}}$ &  1/2     &  80 463.2        & 3.37 \\
      \civ                 &  1548.187  &  8.0083     &  64.4935      &  1$s^{2}$2$s$       &   $^{2}$S  & 1/2         & 1$s^{2}$2$p$          &  $^{2}$P$^{{\rm o}}$ &  3/2     &  64 591.7        & 2.65  \\
      \civ                 &  1550.772  &  7.9950     &  64.4935      & 1$s^{2}$2$s$        &  $^{2}$S   & 1/2         & 1$s^{2}$2$p$          &  $^{2}$P$^{{\rm o}}$ &  1/2     &   64 484.0       & 2.64  \\
      \heii               &   1640.474  &  7.5578     & 54.4178       & 2$p$ 	      &  $^{2}$P$^{{\rm o}}$ &  3/2  &  3$d$ 	                 & $^2$D                  &  5/2        &  60 958.0       & 10.35 \\
      \heii               &   1640.490  &  7.5578     & 54.4178       & 2$p$ 	      &  $^{2}$P$^{{\rm o}}$ &  3/2  &  3$d$ 	                 & $^2$D                  &  3/2        &  60 957.4       & 1.73 \\
      \ciii                 &  1906.683  &  6.5026     & 47.8878       & 1$s^{2}$2$s^{2}$   &   $^{1}$S   & 0            & 1$s^{2}$2$s$2$p$  &  $^{3}$P$^{{\rm o}}$ &   2        &   52 447.1       & 5.19$\times10^{-11}$ \\
      \ciii                 &  1908.734  &  6.4956     & 47.8878       & 1$s^{2}$2$s^{2}$   &  $^{1}$S    & 0            & 1$s^{2}$2$s$2$p$  &  $^{3}$P$^{{\rm o}}$ &  1         &  52 390.8        & 1.14$\times10^{-6}$  \\
       \mgii              &  2795.528  &  4.4338     & 15.0353       & 2$p^{6}$3$s$        &  $^{2}$S    & 1/2        & 2$p^{6}$3$p$          &  $^{2}$P$^{{\rm o}}$ &   3/2     &  35 760.9       & 2.60  \\
      \mgii               &  2802.705  &  4.4224     & 15.0353       & 2$p^{6}$3$s$        &  $^{2}$S    & 1/2        & 2$p^{6}$3$p$          &  $^{2}$P$^{{\rm o}}$ &   1/2     &  35 669.3       & 2.57 \\
      H Ba $\beta$   &  4861.333  &  2.5497     & 13.5984       & 2                              &                 &               & 4                             &                                 &              &  20 564.8       & 0.0842  \\
      \hline   
      \hline
    \end{tabular}
    \caption{
      Strong UV/optical spectral emission lines in quasars, and their
      atomic data.  Data from the NIST Atomic Spectra Database
      \citep{Kramida2018, Kramida2019}.  The Transition Energies are
      $E=hc/\lambda$ for the given wavelength. The Ionization Energy is the
      energy required to ionize the given species, e.g. 64.49 eV are needed
      to create a \cv\ ion.  Transitions Level configurations are given in
      standard spectroscopic notation $A_{i,j}$ are transition
      probabilities.  Data from the NIST Atomic Spectra Database
      \citep{Kramida2018, Kramida2019}.}
    \label{tab:atomic_lines}
  \end{centering}
\end{table*}

CLQs to date have primarily been defined according to the
(recombination) Balmer emission line properties with particular
attention paid to the H$\beta$ emission line, observed from optical
spectroscopy. Recent work report on discoveries of \mgii Changing-look
AGN \citep{Guo2019, Homan2019}. However, current CLQ studies have
primarily been at redshifts $z<1$.

While there have been many studies on triply ionized carbon, i.e
\civ, these have tended to focus on broad absorption line quasars
\citep[BAL quasars; see Table 1 of][]{Hemler2019} or the Baldwin Effect
\citep[BEff; ][]{Baldwin1977, Bian2012, Jensen2016,
Hamann2017}\footnote{As noted in \citet{Rakic2017}, two different
types of Baldwin effect are present in the literature: the {\it
global} (or {\it ensemble}) Baldwin Effect, which is an
anti-correlation between the Equivalent Width (EW) of the emission line and the
underlying continuum luminosity of {\it single-epoch} observations of
a {\it large number} of AGN and second, the {\it intrinsic} Baldwin
effect, the same anti-correlation but in an {\it individual, variable}
AGN \citep{PoggePeterson1992}.}.  Dramatic changes in the
collisionally excited broad {\it emission} line (BEL) of \civ\ and
indeed \ciii have not to this point been reported.

\begin{table*}
  \centering
  \begin{tabular}{l l   r ll  r r l}
    \hline 
    \hline 
    \multirow{2}{*}{Object} & \multirow{2}{*}{Redshift} & $g$-band & \multirow{2}{*}{MJD} & \multirow{3}{*}{Instrument} & Exposure      &   SDSS               & \multirow{2}{*}{Notes} \\
                                         &                                        &   (mag)      &                                 &                                             &  Time (secs) &   Plate-FiberID  & \\
    \hline  
                                       & 2.068                              & 18.27          &  53498                   & SDSS                                     &  8057            & 2089-427             & \\
 J120544.7+342252.4     & 2.071                             &                    &  58538                   & DBSP                                     &  1800            &  ---                      &  Average conditions \\
                                       & 2.071                             &                     &  58693                  & DBSP                                      &  2400            &  ---                      &          \\
                                       &                                       &                     &                               &                                              &                      &                               &                               \\
                                      & 2.185                              & 19.77           &  54553                   & SDSS                                     &   4801            &  2948-614              & \\
J163852.9+282707.7     & 2.186                              &                    &  55832                     & BOSS                                    &   3600            &  5201-178            & \\
                                      & 2.182                              &                     &  58583                     & LRIS                                    &  1800              &  ---                      & \\
                                      &                                         &                    &                                 &                                            &                      &                              &                                 \\
                                      & 2.217                               & 19.97          &  56189                    &  BOSS                                    &  2700            &   6118-720          & \\
 J222818.7+220102.9   & 2.222                               &                    &  56960                     & BOSS                                    &  4500            &   7582-790          & eBOSS reobservation \\ 
                                     &  2.222                               &                    &  58693                    &  DBSP                                    & 2400             &    ---                        &    \\
    \hline \hline   
  \end{tabular}
  \caption{Details of our spectroscopic observations.  Redshift errors are
    typically $\pm$0.002.  SDSS, BOSS and eBOSS spectra have
    $\mathcal{R}\sim$2,000.  DBSP: Double Spectrograph on the Palomar
    200-inch telescope.  LRIS: Low Resolution Imaging Spectrometer on Keck
    I 10m telescope.} 
  \label{tab:obs_notes}
\end{table*}

Here, we present new results for three quasars which show dramatic
changes in their \civ\ and \ciii broad emission line properties as
well as in their underlying continuum. These are some of the first
examples of ``Changing Look Quasars'' at high ($z>1$)
redshift. Moreover, these are the first cases for substantial changes
of ions with high ionization potentials (I.P.'s $>$2 Rydberg), thus
linking the ionizing photons to the energetic inner accretion disk,
potentially by inverse Compton scattering of lower energy photons to
higher energies.

Details of the atomic transitions that produce strong rest-frame
UV/optical lines in quasars are given in
Table~\ref{tab:atomic_lines}. In this paper we use the wavelengths of
1548.202 and 1550.774~\AA\ for the \civ\ doublet
\citep{Kramida2018}. For ionisation energies, 47.89 eV (3.519 Ry) is
required for doubly-ionised C~III to become triply-ionised \civ.
64.49 eV (4.74 Ry) is the energy needed to ionize \civ\ itself. This
energy corresponds to a thermal temperature of $T \gtrsim$
4$\times10^{5}$, implying a heating energy source of (soft) X-ray
photons. 

\citet{Wilhite2006} examine \civ\ variability in a sample
of 105 quasars observed at multiple epochs by the Sloan Digital Sky
Survey \citep[SDSS;][]{York2000, Stoughton2002, Abazajian2009}.  They
find a strong correlation between the change in the \civ\ line
flux and the change in the line width, but no correlations between the
change in flux and changes in line center and skewness.  These authors
find that the relation between line flux change and line width change
is consistent with a model in which a broad line base varies with
greater amplitude than the line core. The \civ\ lines in these
high-luminosity quasars appear to be less responsive to continuum
variations than those in lower luminosity AGN.

\citet{Richards2011} explored the BEL region in over 30,000 $z > 1.54$
SDSS quasars, concentrating on the properties of the \civ\ emission
line. These authors consider two well-known effects involving the
\civ\ emission line: {\it (i)} the anti-correlation between the \civ\
equivalent widths (EWs) and luminosity (i.e., the Baldwin Effect;
BEff) and {\it (ii)} the blueshifting of the peak of \civ\ emission
with respect to the systemic redshift.  We denote the velocity offset
of emission lines as $V_{\rm off}$ and use the convention that a
positive $V_{\rm off}$ value means the line is blueshifted while a
negative $V_{\rm off}$ value means the line is redshifted.
\citet{Richards2011} find the blueshift of the \civ\ emission line, is
found to be nearly ubiquitous, with a mean shift of $V_{\rm off}
\sim$810 km s$^{-1}$ for radio-quiet (RQ) quasars and $V_{\rm off}
\sim$360 km s$^{-1}$ for radio-loud (RL) objects. \citet{Richards2011}
also find the BEff is present in both the RQ and RL studied samples.
These author conclude that these two \civ\ parameters (EQW and
blueshift) are capturing an important trade-off between ``disk'' and
``wind'' components in the disk-wind model of accretion disks
\citep[e.g.,][]{Murray1995, Elvis2000, Proga2000, Leighly2004b}, with
one dominating over the other depending on the shape of the SED.

Using the multi-epoch spectra of 362 quasars from the Sloan Digital
Sky Survey Reverberation Mapping \citep[SDSS-RM; ][]{Shen2015,
Shen2019} project, \citet{Sun2018} investigate the blueshift of \civ\
emission relative to \mgii emission, and its dependence on quasar
properties. These authors confirm that high-blueshift sources tend to
have low \civ\ EWs, and that the low-EW sources span a range of
blueshift. Other high-ionization lines, such as \heii, also show
similar blueshift properties. The ratio of the line width of \civ\ to
that of \mgii increases with blueshift. \citet{Sun2018} also find that
quasar variability might slightly enhance the connection between the
\civ\ blueshift and EW, though further investigation here is
warranted. There is also the finding that the objects with that
largest blueshifts are less variable and tend to have higher Eddington
ratios. \citet{Sun2018} explain their results these by suggesting that
quasar SEDs have weaker X-ray emission (or at least a larger UV/optical-to-Xray
spectral index, $\alpha_{\rm ox}$) and thus become {\it
softer} with {\it increasing Eddington ratio} along with the presence
of X-ray shielding by the inner accretion disk. However, a high
Eddington ratio alone might be an insufficient condition for the \civ\
blueshift. Recent investigations also include \citet{Meyer2019} and
\citet{Doan2019}. \citet{Dyer2019} provide a detailed analysis of 340
quasars at high-redshift ($1.62<z<3.30$) from the SDSS-RM project,
which we will compare our results to.

%%    P a p e r    O v e r v i e w
The purpose of this paper is, for the first time, to systematically
access and report on the CLQ phenomenon at high ($z>2$)
redshift. While accessing this phenomenon at an earlier cosmic epoch
is somewhat interesting, the main value of this study is we move from
the low-ionization energy Balmer emission line series to the
high-ionization emission lines, in particular \civ\ $\lambda$1549.

This paper is organised as follows. In Section 2, we describe our
sample selection, catalogues, and observational data sets.  In Section
3, we show the high-$z$ quasars and report the line properties for the
quasars at the observed epochs.  We give a very brief theoretical
discussion in Section 4. We present our conclusions in Section 5.  We
report all magnitudes on the AB zero-point system \citep{Oke_Gunn1983,
Fukugita1996} unless otherwise stated. For the WISE bands, $m_{\rm AB}
= m_{\rm Vega} + m$ where $m = (2.699, 3.339)$ for WISE W1 at
3.4$\mu$m and WISE W2 at 4.6$\mu$m, respectively \citep{Cutri2011,
Cutri2013}. We adopt a flat $\Lambda$CDM cosmology with
$\Omega_{\Lambda} = 0.73 $, $\Omega_{\rm M} = 0.27$, and $h =
0.71$. All logarithms are to the base 10.


%%%%%%%%%%%%%%%%%%%%%%%%%%%%%%%%%%%%%%%%%%%%%%%%%%%%%%%%%%%%%%%%%%%%%%%%%%%%%
%%%%%%%%%%%%%%%%%%%%%%%%%%%%%%%%%%%%%%%%%%%%%%%%%%%%%%%%%%%%%%%%%%%%%%%%%%%%%
%%
%%    SECTION 2   SECTION 2   SECTION 2   SECTION 2   SECTION 2   SECTION 2  
%%    SECTION 2   SECTION 2   SECTION 2   SECTION 2   SECTION 2   SECTION 2  
%%    SECTION 2   SECTION 2   SECTION 2   SECTION 2   SECTION 2   SECTION 2  
%%
%%%%%%%%%%%%%%%%%%%%%%%%%%%%%%%%%%%%%%%%%%%%%%%%%%%%%%%%%%%%%%%%%%%%%%%%%%%%%
%%%%%%%%%%%%%%%%%%%%%%%%%%%%%%%%%%%%%%%%%%%%%%%%%%%%%%%%%%%%%%%%%%%%%%%%%%%%%
\section{Data, CLQ Selection and Line measurements}
In this section we present the photometric data used to select the
CLQs, and then give details to the multiwavelength data where we have
it. We then give details of the spectroscopic data including emission
lines measurements.

\subsection{Photometry}
\subsubsection{Optical Photometry}
We use optical data from the Catalina Real-time Transient Survey
\citep[CRTS;][]{Drake2009, Mahabal2011}, the Panoramic Survey
Telescope and Rapid Response System \citep[PanSTARRS;][]{Kaiser2010,
Stubbs2010, Tonry2012, Magnier2013} and the Zwicky Transient Facility
\citep[ZTF;][]{Bellm2019_ZTFOverview}. 

The CRTS archive\footnote{http://catalinadata.org} contains the
Catalina Sky Survey data streams from three telescopes -- the 0.7 m
Catalina Sky Survey (CSS) Schmidt and 1.5 m Mount Lemmon Survey (MLS)
telescopes in Arizona and the 0.5 m Siding Springs Survey (SSS)
Schmidt in Australia. CRTS covers up to $\sim$2500 deg$^2$ per night,
with 4 exposures per visit, separated by 10 min. The survey observes
over 21 nights per lunation. The data are broadly calibrated to
Johnson $V$ \citep[see ][for details]{Drake2013} and the current CRTS
data set contains time series for approximately 400 million sources to
$V \sim 20$ above Dec $> -30$ from 2003 to 2016 May (observed with CSS
and MLS) and 100 million sources to $V \sim 19$ in the southern sky
from 2005 to 2013 (from SSS). CRTS has been used to study distant
quasars previously \citep{Graham2014, Graham2015, Graham2015Nature,
Graham2017, Graham2019b}.

PanSTARRS data is obtained via the Pan-STARRS Catalog Search
interface\footnote{https://catalogs.mast.stsci.edu/panstarrs/}.  We
query the PS1 DR2 Detection catalog.

The ZTF is a new robotic time-domain sky survey capable of visiting
the entire visible sky north of -30 deg. declination every night. ZTF
observes the sky in the $g$, $r$, and $i$-bands at different cadences
depending on the scientific program and sky region
\citep{Bellm2019_ZTFSurveys, Graham2019_ZTF}. The ZTF 576 megapixel
camera with a 47 deg$^{2}$ field of view, installed on the Samuel
Oschin 48-inch Schmidt Telescope, can scan more than 3750 deg$^{2}$
per hour, to a 5$\sigma$ detection limit of 20.7 mag in the $r$-band
with a 30sec exposure during new moon \citep{Masci2019}.
\begin{figure*}
  \centering
  %% trim=l b r t
  \includegraphics[width=16.7cm, trim=0.3cm 0.05cm 0.45cm 0.1cm, clip]
  {figures/J1205+3422_landscape_20191112.png}
  \includegraphics[width=16.7cm, trim=0.3cm 0.05cm 0.40cm 0.1cm, clip]
  {figures/J1638+2827_landscape_20191112.png}
  \includegraphics[width=16.7cm, trim=0.3cm 0.0cm  0.35cm 0.1cm, clip]
  {figures/J2228+2201_landscape_20191112.png}
  \vspace{-12pt}
  \caption[]{The three high-$z$ CLQ quasars; 
    J1205+3422 (top), 
    J1638+2827 (middle) and  
    J2228+2201 (bottom). 
    The light curve data is present in the panels on the left hand side, with the 
    spectral epoch observational timings indicated by vertical lines. 
    The spectra are on the right hand side, with zoom-in's on the Ly$\alpha$-\nv 
    complex, the \civ\ line, and the \mgii line.}
  \label{fig:civ_clqs}
\end{figure*}

\subsection{Multi-Wavelength Properties}
{\bf DS to write something on FIRST non-detections; possibly check the X-rays too?} \\
Mid-infrared data (3.4 and 4.6$\mu$m) is available from the beginning
of the {\it Wide-field Infrared Survey Explorer} ({\it WISE}) mission
\citep[2010 January; ][]{Wright2010} through the fifth-year of {\it
NEOWISE-R} operations \citep[2018 December; ][]{Mainzer2011}. The {\it
WISE}WISE scan pattern leads to coverage of the full-sky approximately
once every six months (a ``sky pass''), but the satellite was placed
in hibernation in 2011 February and then reactivated in 2013
October. Hence, our light curves have a cadence of 6 months with a
32-month sampling gap.


\subsection{CLQ Selection}
{\bf MJG to finalise this} \\
Our high-$z$ CLQs were identified as follows.  We selected all 64,774
SDSS DR15 sources with $z > 0.35$ classified as `quasar', having at least
two spectra separated by at least 100 days, and with a corresponding
CRTS light curve. We fitted a damped random walk to the CRTS data via
Gaussian process regression and the photometric magnitudes at the
epochs of the SDSS spectra for a given source are predicted. Those
where $|\Delta V| > 0.3$ are then selected for visual
inspection. Only three quasars SDSS J120544.7+342252.4 (hereafter
J1205+3422), SDSS J163852.93+282707.7 (hereafter J1638+2827) and SDSS
J222818.76+220102.9 (hereafter J2228+2201), satisfied these selection
criteria and showed interesting or dramatic emission line behaviour.


%% From Section 3.1; Placed here to help with overall formatting of the paper
\begin{table*}
  \small
  \begin{centering}
    \begin{tabular}{l  l | c c | c c c c | c}
      \hline
      \hline
                        &                   &  \multicolumn{2}{c}{Cont. @ 1450\AA }                                           &   \multicolumn{4}{c}{CIV 1549\AA}                                                                          &  Virial product \\
       Object      &   MJD           &       $\nu L_{\nu}$             &         Slope                                               &   Luminosity                &     FWHM               &    V$_{\rm off}$          &          EW        &  log($\nu L_{\nu})^{0.5} $ \\
                       &                     & $/10^{42}$ erg s$^{-1}$ & (F$_\lambda \propto \lambda^\alpha$)    & $/10^{42}$ erg s$^{-1}$ &    / km s$^{-1}$ &  /  km s$^{-1}$         &     /  \AA   & $\times$ FWHM$^{2}$   \\
      \hline
                       &  53498          &   41630   $\pm$   40    &  -1.70 $\pm$ 0.01                               &   700    $\pm$ 49       &  4700 $\pm$  120     &     580    $\pm$   80      &   27  $\pm$  1      &  9.66 $\pm$ 0.02\\
 J1205+3422  &  58538$^*$   &    2870    $\pm$   40     &  -1.29 $\pm$ 0.09                              &  170    $\pm$ 30      & 14900 $\pm$ 2800   &     980    $\pm$  690      &   92    $\pm$    16        & 10.08  $\pm$ 0.14 \\
                       &  58693          &    9050    $\pm$   10     &  -1.46 $\pm$ 0.00                              &  400   $\pm$   5       &  6980 $\pm$   90      &    1080    $\pm$   30      &   70 $\pm$  1  &  9.67 $\pm$ 0.01\\
                       &                      &                                      &                                                             &                                  &                                   &                                       &                                   &                  \\
                       &  54553         &    4060    $\pm$  120    &  -1.4  $\pm$ 0.2                                 & 330      $\pm$ 10      &  4630 $\pm$  190     &     180    $\pm$   60     &  128  $\pm$  4.4   &  9.14 $\pm$ 0.04\\
 J1638+2827   &  55832        &    2570    $\pm$   20      &  -2.16 $\pm$ 0.06                             & 100    $\pm$  5        &  4990 $\pm$  300      &     190    $\pm$   90      &   64  $\pm$  3.2   &  9.10 $\pm$ 0.05\\
                       &  58583        &    8600    $\pm$   20      &  -2.01 $\pm$ 0.01                             & 420    $\pm$  5        &  4620 $\pm$   70        &     100    $\pm$   20      &   81 $\pm$  0.93  &  9.30 $\pm$ 0.01\\
                       &                    &                                         &                                                          &                                   &                                   &                                        &                                    &                  \\
                       &  56189$^*$ &     860    $\pm$   40     &  -0.6  $\pm$ 0.1                                  &  40    $\pm$  10       &  5930 $\pm$  990      &     450    $\pm$  290      &   73  $\pm$  9.8   &  9.01  $\pm$ 0.13 \\
 J2228+2201  &  56960        &    9150    $\pm$   30     &  -1.88 $\pm$ 0.02                               &  340    $\pm$  10     &  7000 $\pm$  200     &    -255    $\pm$   60      &   60  $\pm$  1.4   &  9.67 $\pm$ 0.03\\
                      &  58693        &    2810  $\pm$    5        &  -1.38 $\pm$ 0.01                               &  160    $\pm$   5      &  5930 $\pm$   80       &     180    $\pm$   30      &   91 $\pm$  0.95  &  9.27 $\pm$ 0.01\\
      \hline
      \hline
      \hline
    \end{tabular}
    \caption{
      Continuum at 1450\AA\ and \civ\ spectral measurements for the
      three quasar considered in this work, at all observation epochs, as
      calculated by QSFit.  For the emission-line velocity offsets, a
      positive value means the line is blueshifted.  The last column shows
      the virial product calculated as $(\nu L_{\nu})^{0.5} \times {\rm
        FWHM}^2$.  {\bf The last column shows the virial product calculated as
        $(\nu L_{\nu})^{0.5} \times {\rm FWHM}^2$, and its uncertainty
        (calculated by propagating uncertainties on both $\nu L_{\nu}$ and the
        FWHM).  The virial products are calculated to check if the BH mass
        varies significantly across epoch.  } $^{*}$The \civ\ line is very
      faint (with respect to the continuum), and the associated estimates
      are likely unreliable.
    }
    \label{tab:QSFit-results}
  \end{centering}
\end{table*}


\subsection{Spectroscopy}
An overview of our spectroscopic observations is given in
Table~\ref{tab:obs_notes}.  The spectra are from the SDSS
\citep{Stoughton2002, DR7, Schneider2010}, the SDSS-III Baryon
Oscillation Spectroscopic Survey \citep[BOSS; ][]{Eisenstein2011,
Dawson2013, Smee2013, Alam2015, Paris2017} and the SDSS-IV Extended
Baryon Oscillation Spectroscopic Survey \citep[eBOSS; ][]{Dawson2016,
Abolfathi2018, Paris2018}.  These quasars were targetted via a range
of techniques and algorithms \citep[see ][]{Richards2002, Ross2012,
Myers2015}. The SDSS, BOSS, and eBOSS data are supplemented by spectra
from the Low Resolution Imaging Spectrometer \citep[LRIS; ][]{Oke1995}
on the 10m Keck {\sc I} telescope.  and the Double Spectrograph (DBSP)
instrument on the 200'' Palomar telescope.

\subsubsection{Spectrophotometry}
{\bf MJG to check/finalise} \\
We want all our spectra to have reliable flux calibrations.  SDSS
spectra ae spectrophotometrically calibrated.  BOSS and eBOSS spectra
have spectrophotometric corrections applied in the latest data release
\citep{Hutchinson2016, Jensen2016, Margala2016}.  Due to the high-$z$
of our objects, \oiii is not available to us to use as a calibrating
flux line. Instead we use photometric data from the ZTF since all our
non-SDSS/BOSS/eBOSS data are taken after MJD 57500.

\subsection{Emission Line and Power-law Slope Measurements}
We use the measured quasar emission line properties from several catalogues: 
\citet{Shen2011}, \citet{Hamann2017}, \citet{Kozlowski2017}, and
\citet{Calderone2017}.

In particular we use the Quasar Spectral Fitting (QSFit) software
package presented in \citet{Calderone2017}. This provides luminosity
estimates as well as width, velocity offset and equivalent width of 20
emission lines, including \civ, \ciii and \mgii.  We process and fit
all nine spectra using the lastest version (v1.3.0) of the QSFit
\href{https://qsfit.inaf.it/cat_1.30/onlinefit.php}{online
calculator}. The host galaxy and blended iron emission at rest-frame
optical wavelengths components are automatically disabled when they
can not be constrained by the available data, such as the case for all
our objects (we do not have infrared spectral data).  Power-law
continuum slopes, $\alpha$, where $f_{\lambda} \propto
\lambda^{\alpha}$, are also reported in these catalogues and from
QSFit.



%%%%%%%%%%%%%%%%%%%%%%%%%%%%%%%%%%%%%%%%%%%%%%%%%%%%%%%%%%%%%%%%%%%%%%%%%%%%%
%%%%%%%%%%%%%%%%%%%%%%%%%%%%%%%%%%%%%%%%%%%%%%%%%%%%%%%%%%%%%%%%%%%%%%%%%%%%%
%%
%%   SECTION 3   SECTION 3   SECTION 3   SECTION 3   SECTION 3   SECTION 3  
%%   SECTION 3   SECTION 3   SECTION 3   SECTION 3   SECTION 3   SECTION 3  
%%   SECTION 3   SECTION 3   SECTION 3   SECTION 3   SECTION 3   SECTION 3  
%%
%%%%%%%%%%%%%%%%%%%%%%%%%%%%%%%%%%%%%%%%%%%%%%%%%%%%%%%%%%%%%%%%%%%%%%%%%%%%%
%%%%%%%%%%%%%%%%%%%%%%%%%%%%%%%%%%%%%%%%%%%%%%%%%%%%%%%%%%%%%%%%%%%%%%%%%%%%%
\section{Results}
\subsection{Photometric and Overal Spectral Evolution}
Figure~\ref{fig:civ_clqs} presents the optical and mid-IR light
curves for three high-$z$ CLQ quasars.  Figure~\ref{fig:civ_clqs} also
shows the spectra for each epoch, with the MJD of observation given by
the dashed vertical lines in the light curves.

For J1205+3422, our spectral observations cover 5195 days observed,
1691 days in the rest-frame. This quasar was initially identified in
SDSS in 2005 May, as a bright, $g\approx18.0$, blue-sloped quasar with
broad Si\,{\sc iv}, \civ, \ciii and Mg\,{\sc ii}. \ciii and \civ\ have large
blueshifts of $\approx$2600$\pm$150 and $\approx$1150$\pm$100 km
s$^{-1}$, respectively.  By 2019, however, the optical brightness
dropped by $\sim$1.5 magnitudes and the spectra are significantly less
blue.  While \lya and \nv are detectable in both 2019 spectra, \civ\
has all but disappeared in the 2019 June spectrum.  The broad \ciii
emission has disappeared between the 2005 and 2019 spectra. The
changes in \civ\ and \ciii going from broad emission to barely
detectable have on the timescales of $\approx$50 days in the
rest-frame.

For J1638+2827, our spectral observations cover 4030 days observed,
1265 days in the rest-frame. Here, in the initial epoch spectrum,
\civ\ is broad and bright, as is C\,{\sc iii}. However, $\approx$400
rest-frame days later, the broad \civ\ and \ciii BEL have faded, the
continuum slope around 1400\AA\ has changed from $\approx-1.48$ to
$\approx-2.25$, but the \lya/\nv emission complex is very similar in
shape and line flux intensity. Around 870 days in the rest-frame after
the second spectral epoch, Ly$\alpha$, N\,{\sc v}, \civ, C\,{\sc iii}] and \mgii are all
apparent and broad, with \mgii being seen for the first time at high
signal-to-noise. The light curve is consistent with this spectral
brightening, increasing from $\sim$20th magnitude to $\sim$18.5th 
magnitude at optical wavelengths. An absorption feature between \lya
and \nv is seen in all three spectral epochs.

For J2228+2201, our spectral observations cover 2504 days observed,
778 days in the rest-frame. Over the course of 240 rest-frame days,
\civ\ and \ciii both {\it emerge} as BELs and the standard UV/blue
continuum slope increases in flux.  Then, over the course of 538 days
in the rest-frame, the broadline emission, while still very present,
reduces in line flux the UV/blue continuum diminishes, though is still
more luminous than the initial BOSS spectrum.


%% trim={<left> <lower> <right> <upper>}
\begin{figure}
  \centering
  \includegraphics[width=8.40cm, trim=0.37cm 0.3cm 0.0cm 0.2cm, clip]{figures/CIVregions.png}
    \vspace{-18pt}
  \caption{Observed spectra and best fit model in the region relevant
    to \civ\ emission line for the quasars considered here.
    The solid blue line is the Lorentzian profile fit the to the \civ\ emission line.
    The long-dashed red line is the continuum fit, with the solid orange
    line giving the overall fit. The short-dashed vertical lines gives the
    rest wavelengths of the \civ\ doublet. }
  \label{fig:QSFit-CIV}
\end{figure}
\subsection{\civ\ Emission Line Evolution}
We analyzed the spectra of the three quasars, at all observational
epochs, using the QSFit spectral fitting package
\citep{Calderone2017}.  One advantage of using QSFit is that it allows
constraints on the slope and luminosity of the broad band continuum of
the source. The relevant estimated quantities, including continuum
luminosity and slope at rest-frame 1450\AA, \civ\ line luminosity,
FWHM and EW are given in Table~\ref{tab:QSFit-results}. All fits are
performed with $E(B-V) = 0$ and the best fit model in the region of
the \civ\ emission line are shown in Figure~\ref{fig:QSFit-CIV}.

All \civ\ lines are fitted with a single, broad, Lorentzian profile.
This allows us to account for the narrow peak of the \civ\ line.  No
narrow components are considered for several reasons: in the epochs of
highest brightness the ``narrow'' component would have FWHM~$\sim2-3
\times 10^{3}$~km~s$^{-1}$, i.e. values exceeding the usual widths of
genuine narrow lines ($\lesssim$10$^{3}$ km s$^{-1}$); by allowing a
second component, to have such large widths their parameters would
become highly degenerate with the ``broad'' components, and the latter
would also have much larger FWHM ($\sim$$10^{4}$ km s$^{-1}$); and by
neglecting the narrow components we have a consistent fit across all
epochs.

The quasar continuum, evaluated at 1450\AA\ , and the \civ\ line
luminosities follow a similar evolution, with a ratio of $\sim$20-30,
confirming that the main driver for emission line variability is
likely the broad band continuum itself.  For all sources except
J1638+2827, the slope of the continuum changes with luminosity
following a ``bluer-when-brighter'' pattern, suggesting that a
distinct emerging component is responsible or both the slope and
luminosity variations.  In J1638+2827 the opposite behaviour is
observed, especially in the first observation epoch.  However, this
may be a bias due to the limited wavelength range available which
extends to $\lambda \sim 1240$\AA\ for the first epoch, while it extend
to shorter wavelengths for the other epochs (respectively 1140\AA\ and
1010\AA).  The latter suggests that the ``emerging'' component is more
prominent at UV wavelengths, and a sufficient wavelength coverage is
required to detect it. We find there is no need for a narrow component
in J2228+2201 MJD 56189.  This is because the data resemble a P-Cygni
profile, with blueshifted absorption. This feature is very narrow, and
the uncertainties large, such that the overall $\chi^{2}$ is only
marginally affected by the addition of a further component. We
therefore avoid using a narrow component for this spectrum only, since
that would be inconsistent with the other analyses.
In all cases where the \civ\ line profile is reliably constrained the
\civ\ FWHM is approximately constant, with maximum variations
$\lesssim$~1000 km s$^{-1}$, despite the significantly larger
variations in the line luminosities.


\begin{figure}
  \centering
  \includegraphics[width=8.5cm, trim=0.2cm 0.2cm 0.0cm 0.2cm, clip]
  {figures/CIV_CLQs_MBHvsREW_QSFit20K.png}
   \vspace{-12pt}
   \caption[]{The virial black hole masses of $\approx$20,000 $z>1.5$ quasars
     from QSFit catalogue \citep{Calderone2017} and the \civ\
     Equivalent widths.  Different quasars are given different colours:
     J1205+3422 purple; J1638+2827 green; J2228+2201 is blue.  Different
     epochs different symbols: first epoch circles; second epoch squares;
     third epoch diamonds.  As noted in Table~\ref{tab:QSFit-results},
     $^{*}$ signifies the \civ\ line is very weak and the associated
     estimates are likely unreliable.}
   \label{fig:CIV_MBHvsREW}
\end{figure}
\subsection{Virial Black Hole Masses}\label{sec:BH_masses} 
The FWHM of broad lines is likely related to the mass of the
supermassive black hole powering the quasar phenomenon, which is
assumed to be constant on any human timescale.  Hence it is
instructive to check whether the virial product, which is the basic
quantity used to calculate the single epoch black hole mass estimate,
show any variation.

Using the estimates for the continuum luminosity and FWHW from the
single-epoch spectra, one can estimate the central black hole mass
\citep[e.g.][]{Shen2011, Calderone2017}.  This approach assumes that
the broad-line region (BLR) is virialized, the continuum luminosity is
used as a proxy for the BLR radius, and the broad line width (FWHM) is
used as a proxy for the virial velocity.  This ``virial mass''
estimate can then be expressed as:
\begin{equation}
  \log(M_{\rm BH}) = (\nu L_{\nu})^{\gamma} \times ({\rm FWHM})^{\delta}
\end{equation}
where $\gamma=0.5$ and $\delta=2$.  The virial product for the \civ\
CLQs is reported in the last column of Tab.~\ref{tab:QSFit-results}. 
The uncertainty typically associated to the single epoch mass estimate
is $\sim$0.5 dex, hence the virial product at all epochs are
remarkably constant and compatible with a single value of black hole
mass for each source, even in those cases where the \civ\ estimates
are possibly unreliable. The object showing larger variation is
J2228+2201, although the extreme values span a range of 0.48 dex.

\citet{Shen2011} reports Virial BH mass based on \mgii and \civ\ as
$\log (M_{\rm BH, Vir}/M_{\odot}) = 9.55\pm0.05$ and $\log(M_{\rm BH, Vir}/M_{\odot}) =
9.49\pm0.04$, respectively. We thus use the mean of these five values
(three from our three epochs and the two \citet{Shen2011} values) to
establish a working mean SMBH mass of $\log (M_{\rm BM, vir}/M_{\odot})
= 9.726$ for J1205+3422.  J1638+2827 also has two virial mass
measurments from Shen2011, but also has two measurements from
\citet{Kozlowski2017} (again using \civ and \mgii). The mean mass
measurement for J1638+2827 here is $\log(M_{\rm BM, vir}/M_{\odot} =
9.091$. For J2228+2201 there are two \citet{Kozlowski2017}, leading to
a mean mass of $\log(M_{\rm BM, vir}/M_{\odot} = 9.366$. We use these
mean SMBH masses when calculating the Eddington ratios.

From the virial mass estimates, all our objects have SMBH masses $M
>10^{9} M_{\odot}$.  This is at the upper end of SMBH masses at all
epochs, and towards the extreme of the mass distribution for $z\sim2$
objects.  Figure~\ref{fig:CIV_MBHvsREW} shows the \civ\ Equivalent
width and the virial black hole masses for a sample of $\approx$20,000
$z>1.5$ SDSS quasars from the QSFit catalog as well as the estimates
for the three \civ\ CLQs.



\begin{figure}
  \centering
  \includegraphics[width=8.5cm, trim=0.2cm 0.2cm 0.0cm 0.2cm, clip]
  {figures/CIV_CLQs_REWvsFWHM_20191128.png}
  \vspace{-12pt}
  \caption[]{The Rest Equivalent Width (REW) vs. Full Width Half Maximum (FWHM) 
    of the \civ\ emission line in the BOSS DR12 quasar sample using the catalogue 
    of \citet{Hamann2017}.
   Symbols as in Fig.~\ref{fig:CIV_MBHvsREW}.}
  \label{fig:REWvsFWHM}
\end{figure}
\subsection{Quantified Temporal Evolution of \civ\ Emission}
Quasars with interesting physical properties, such as extreme
outflows, can be selected using EW and FWHM measurements, \citep[e.g.,
the ``Extremely Red Quasars'' (ERQs)][]{Ross2015, Zakamska2016,
Hamann2017, Zakamska2019}. Figure~\ref{fig:REWvsFWHM} shows the rest
Equivalent Width (REW) versus the Full Width Half Maximum (FWHM) of
the \civ\ emission line in the BOSS DR12 quasar sample using the
catalogue of \citet{Hamann2017}.

The velocity offsets of the \civ\ line are also
approximately constant, and compatible with a single value (within
3$\sigma$).  The exception is J2228+2201, where a significant change
($\sim$7$\sigma$) is observed between the second and third observation
epochs. The magnitude of the velocity blueshifts ($\sim200-1100$ km s$^{-1}$)
are consistent with recent results into the properties of 
rest-frame UV spectra over the redshift range $1.5 \leq z \leq 7.5$
\citep{Meyer2019}. 


\begin{figure*}
  \centering
  \includegraphics[width=\textwidth]{figures/QSFit-results}
  \vspace{-12pt}
  \caption{Temporal evolution of the spectral properties of the three
    quasar considered in this work.}
  \label{fig:QSFit-results}
\end{figure*}
The temporal evolution of the velocity offsets, along with the
1450\AA\ continuum luminosity and slope, the \civ\ line FWHM
and virial black hole mass are given in Figure~\ref{fig:QSFit-results}.

\begin{table}
  \centering
  \begin{tabular}{l l l r}
    \hline
    \hline
    Object           & MJD      & $L_{\rm bol}$    &  $\eta_{\rm Edd}$  \\
    \hline
                     & 53498    & 47.216           &  -0.610                  \\
J1205+3422 & 58538    &  46.567          &  -1.259  \\                        %% L_Edd (J12)  = 47.82637   for log(M) = 9.726
                     & 58693     &  46.070         &  -1.756 \\
    \hline 
                     & 54553   & 46.166    & -1.025 \\
J1638+2827  & 55832   & 45.981    & -1.210  \\                                            %% L_Edd (J16)  = 47.191370  for log(M) = 9.091
                     & 58583   & 46.504   & -0.687       \\
    \hline 
                      & 56189   & 46.231   &  -1.235 \\
J2228+2201   & 56960    & 47.342   & -0.124 \\                                             %% L_Edd (J22)  = 47.466370  or log(M) = 9.366
                       & 58693   & 46.826     &  -0.640 \\
    \hline
    \hline
  \end{tabular}
  \caption{
    Bolometric luminosities are given in either \citet{Shen2011} or 
    \citet{Kozlowski2017} for the three CLQs. We scale these 
    values using our measured $\nu L_{\nu}$ continuum luminosity values.
    From Section~\ref{sec:BH_masses}, we assume: $\log(M_{\rm BM, vir}/M_{\odot}) = 9.726$ for J1205+3422; 
    $\log(M_{\rm BM, vir}/M_{\odot} = 9.091$ for  J1638+2827, 
    and $\log(M_{\rm BM, vir}/M_{\odot} = 9.366$ for J2228+2201. 
    The Eddington Luminosity is 
    $L_{\rm Edd} = 1.26\times10^{38} (M/M_{\odot})$ erg s$^{-1}$ and
    %%  6.70456406676162      e+47  for J12
    %%  1.5537120898684228  e+47  for J16
    %%  2.9266483634099527  e+47 for J22
   $\eta_{\rm Edd}$ is the log of the Eddington ratio.}
\label{tab:Eddington_ratios} 
\end{table}


\begin{figure}
  \centering
  \includegraphics[width=8.5cm, trim=0.2cm 0.2cm 0.0cm 0.2cm, clip]
  {figures/CIV_CLQs_Baldwin_20191130.png}
   \vspace{-12pt}
   \caption[]{The \civ\ Equivalent width and the underlying continuum luminosity,
     commonly referred to as The Baldwin Plot.
     The continuum luminosities are  from \citet{Calderone2017},
     the REW measurements are Table~\ref{tab:QSFit-results}.
     Symbols as in Fig.~\ref{fig:CIV_MBHvsREW}.
     The dashed red line has slope $\beta=-0.38$.}
  \label{fig:CIV_Baldwin}
\end{figure}
\subsection{The \civ\ Baldwin Effect}
The Baldwin Effect \citep[Beff; ][]{Baldwin1977} is an empirical 
relation between emission-line REWs and continuum luminosity in
quasars \citep[][]{Shields2007, Hamann2017, Calderone2017}.
\citet{Hamann2017} and \citet{Calderone2017} present recent
measurements of the Beff for large quasar samples. 

There is an anti-correlation between the emission-line REWs and
e.g. 1450\AA\ rest continuum luminosity, so that as the underlying UV
continuum luminosity increases, the EW decreases. 
Figure~\ref{fig:CIV_Baldwin} shows this for a sample (from the QSFit catalogue)
for 20,374 quasars.  The slope (not shown) is $\beta$ is $-0.1997$.
This is consistent with \citet{Kozlowski2017}, using their bolometric
luminosity gives a slope of $\beta=-0.251$ and in line with that from
\citet[][ $\beta=-0.23$]{Hamann2017}.

We add the measurements from the three \civ\ CLQ quasars at each epoch
to Figure~\ref{fig:CIV_Baldwin}. We see first that all three quasars
at all three epochs lie on the edge of the $\nu L_{\nu}$-EW
distribution.  Second, with the exception of J1638+2827 on MJD 55832,
all the measurements show an {\it intrinsic Baldwin Effect}
\citep[e.g.][]{Goad2004, Rakic2017}.  The slope of the CLQs intrinsic
BEff is $\approx-0.38$, as shown by the dashed red line in
Fig.~\ref{fig:CIV_Baldwin}.



%%%%%%%%%%%%%%%%%%%%%%%%%%%%%%%%%%%%%%%%%%%%%%%%%%%%%%%%%%%%%%%%%%%%%%%%%%%%%
%%%%%%%%%%%%%%%%%%%%%%%%%%%%%%%%%%%%%%%%%%%%%%%%%%%%%%%%%%%%%%%%%%%%%%%%%%%%%
%%
%%   SECTION 4   SECTION 4   SECTION 4   SECTION 4   SECTION 4   SECTION 4  
%%   SECTION 4   SECTION 4   SECTION 4   SECTION 4   SECTION 4   SECTION 4  
%%   SECTION 4   SECTION 4   SECTION 4   SECTION 4   SECTION 4   SECTION 4  
%%
%%%%%%%%%%%%%%%%%%%%%%%%%%%%%%%%%%%%%%%%%%%%%%%%%%%%%%%%%%%%%%%%%%%%%%%%%%%%%
%%%%%%%%%%%%%%%%%%%%%%%%%%%%%%%%%%%%%%%%%%%%%%%%%%%%%%%%%%%%%%%%%%%%%%%%%%%%%
\section{Discussion}
\subsection{Continuum and Line Changes: Comparisons to recent Observations}
The top row of Figure~\ref{fig:QSFit-results} demonstrates that
both the 1450\AA\ continuum and the \civ\ emission lines can
exhibit large, $>$$\times$10, changes in luminosity, {\it and}
that these continuum-line changes track each other. 

\citet{Trakhtenbrot2019} report on the quasar 1ES 1927+654 which was
initially seenn to lack broad emission lines and line-of-sight
obscuration, i.e, a ``Type 2'' quasar at redshift $z=0.02$. This
object is then seen to spectroscopically change with the appearance of
a blue, featureless continuum, followed by the emergence of broad
Balmer emission lines.  i.e. this quasar changes into a broad-line
Type 1 after a raise in the continuum luminosity.  This suggests that
there is (at least in some cases) a direct relationship between the continuum
and broad emission lines in CLQs.  A similar scenario may have occured
for the 3 high-$z$ quasars presented here, although we lack the
high-cadence multiwavelength, multi-epoch coverage that
\citet{Trakhtenbrot2019} present.  The multiwavelength data that
\citet{Trakhtenbrot2019} have includes a UV spectrum of 1ES 1927+654.
Interestingly, however, there is no evidence for broad UV emission lines, including
\civ, C\,{\sc iii}, or Mg\,{\sc ii}.  The authors attribute the lack of broad UV
emission lines to dust within the BLR, noting that to dust in the
broadline emission region, noting that the continuum emission does not
show any signs of dust extinction.

\citet{MacLeod2019} present a sample of CLQs where the primary
selection requires large-amplitude ($| \Delta g | > 1$ mag, $| \Delta
r | > 0.5$ mag) variability over any of the available time baselines
probed by the SDSS and Pan-STARRS1 surveys. They find 17 new CLQs
which is $\sim$20\% of the observerd sample. This CLQ fraction
increases from 10\% to roughly half, as the continuum flux ratio
between repeat spectra at 3420 \AA\ (rest-frame) increases from 1.5 to
6. \citet{MacLeod2019} note that these candidates are at lower Eddington
ratio relative to the overall quasar population.

Using the same dataset as and extremely variable quasar sample
as \citet{MacLeod2019}, \citet{Homan2019}
investigate the responsiveness of the \mgii broad emission line
doublet in AGN on timescales of several years.  By again focussing on quasars that show large
changes in their optical light-curves, \citet{Homan2019} find that
\mgii clearly does respond to the continuum.  However, a key finding
from \citet{Homan2019} is that the degree of responsivity varies
strikingly from one object to another.  There are cases of \mgii
changing by as much as the continuum, more than the continuum, or very
little at all.  In the majority (72\%) of this highly variable sample,
the behaviour of \mgii corresponds with that of H$\beta$.  However,
there are also examples of \mgii showing variation, but H$\beta$ does
not, and vice versa.

\citet{Graham2019b} report the largest number of H$\beta$ CSQs with 111
sources being identified. \citet{Graham2019b} find that this population
of extreme varying quasars is associated with {\it changes} in the
Eddington ratio, (rather than just the magnitude of the Eddington
ratio itself) and the timescales imply cooling/heating fronts
propagating through the disk.\\
{\bf NPR to finish this off} \\


\subsection{Continuum and Line Changes: Comparisons to theoretical expectaions}
{\bf SF and BM to add a little more here}\\
The \civ\ line is one of the strongest collisionally excited lines in quasar spectra \citep[e.g.][]{HamannFerland1999}, and \civ\ emission probes the photoionization environment produced by the innermost disk, as indicated by RM time-delay measurements. 

In standard \citet{SS73} thin disk models, large changes in the continuum flux are not permitted over short timescales due to the relatively long viscous time associated with such disks. Given the observed short timescale continuum variations, it is not surprising that the \citet{SS73} disk may fail on other fronts. %The \civ\ variations observed in our sources...

This indicates they may comfortably fit into the sample of \civ\ variable quasars explored by \citet{Dyer2019}, and similar to those authors we suggest slim accretion disk models e.g., \citet[][]{Abramowicz1988} or inhomogeneous disk models
\citep[e.g.,][]{DexterAgol2011} may provide viable explanations for our observations.
%% I want to say a bit more here about the generic probe of photoionization vs shielding and conditions in the BL region but that will take more time.


\subsection{Implications for The Baldwin Effect}
The variable properties of the rest-frame UV quasar emission lines have been long studied, with the global (or ensemble) Baldwin Effect (the anti-correlation between the EW of the emission line and the underlying continuum luminosity of single-epoch observations of a large number of AGN, first noted in \citet{Baldwin1977}.  More recently, the intrinsic Baldwin effect, the same anti-correlation but in an individual, variable AGN.  

The X-ray Baldwin Effect \citep[e.g., ][]{Iwasawa_Taniguchi1993}... \citet{Bachev2004} find a 10-fold decrease in EW \civ\ with Eddington ratio (decreasing from $\sim$1 to $\sim$0.01), while \nv shows no change. These trends suggest a luminosity-independent ``Baldwin effect'' in which the physical driver may be the Eddington ratio. \citet{Ge2016} Broad emission lines is a prominent property of Type 1 quasars. 



\begin{figure*}
  \centering
  \includegraphics[width=14.5cm, trim=0.2cm 0.2cm 0.0cm 0.2cm, clip]
  {figures/MJD_vs_Eddington_20191202.png}
  \vspace{-12pt}
  \caption[]{Eddington Ratios of the three \civ\ CLQs.
    Grayscale gives the Eddington Ratio ranges for the SDSS
    \citep[from ][]{Shen2011} and the BOSS \citep[from ][]{Kozlowski2017} 
    catalogues.
    Symbols as in Fig.~\ref{fig:CIV_MBHvsREW}.
    The red region ($L / L_{\rm Edd} \approx 0.02\pm0.01$;  $\eta =-1.7\pm0.13$) is
    the transition accretion rate suggested by \citet{NodaDone2018}.
    A ``High/Soft State'' with $\eta_{\rm Edd} \geq -1$ and a
    a ``Low/Hard State'' with $\eta_{\rm Edd} \leq -2$ are indicated 
    with dahsed lines and arrows. 
  }
  \label{fig:Eddington_ratios}
\end{figure*}
\subsection{Eddington Ratios and State Changes} 
The broad UV and optical lines in quasars are most sensitive to the
extreme ultraviolet (EUV) part of the spectral energy distribution
(SED), with \civ\ (and indeed \heii and \nv) being at the higher
energy end of the EUV distribution.

The soft X-ray excess -- the excess of X-rays below 2 keV with respect
to the extrapolation of the hard X-ray spectral continuum model -- is
a very common feature among type 1 active galactic nuclei (AGN). 
\citet{NodaDone2018} note that
The soft X-ray excess produces most of the ionizing photons, so its
dramatic drop leads to the disappearance of the broad-line region,
driving the ``changing-look'' phenomena.  major difference is that
radiation pressure should be much more important in AGNs, so that the
sound speed is much faster than expected from the gas temperature.
%%
This spectral hardening appears similar to the soft-to-hard state
transition in black hole binaries at $L / L_{\rm Edd} \sim 0.02$
(i.e. $\eta_{\rm Edd} \sim -1.7$), where the inner disc evaporates into
an advection dominated accretion flow, while the overall drop in
luminosity appears consistent with the hydrogen ionization disc
instability.  Crucially \citet{NodaDone2018} make the prediction that
all changing-look AGNs are similarly associated with the state
transition at $L / L_{\rm Edd} \sim$a few per cent.

By comparing the observed correlations between the UV/optical-to-X-ray
spectral index ($\alpha_{\rm ox}$) and Eddington ratio in AGN to those
predicted from observations of X-ray binary outbursts,
\citet{Ruan2019a} find a remarkable similarity to accretion state transitions in prototypical
X-ray binary outbursts, including an inversion of this correlation at
a critical Eddington ratio of $\sim$10$^{-2}$ (i.e. at the same ratio
as motivated by \citet{NodaDone2018}).  These results suggest that the
structures of black hole accretion flows directly scale across a
factor of $\sim10^{8}$ in black hole mass and across different
accretion states. Using \citet{Ruan2019a} as a guide, there are potentially three accretion regimes: 
(1) a ``High/Soft State'' with $\eta_{\rm Edd} \gtrsim -1$; 
(1) a ``Low/Hard State'' with $-2 \lesssim \eta_{\rm Edd} \lesssim  -1$;
(1) a ``Low/Hard State'' with $\eta_{\rm Edd} \lesssim -2$.
These are given as shaded regions in Figure~\ref{fig:Eddington_ratios}.

Building on a previous work \citep[e.g.,][]{JiangYF2014, JiangYF2016, JiangYF2019}, \citet{JiangYF2019arXiv} use global three dimensional radiation magneto-hydrodynamic simulations to study the properties of inner regions of accretion disks around a $5 \times 10^{8}$ M$_{\odot}$ black hole with mass accretion rates reaching 7\% and 20\% of the Eddington value.




%%%%%%%%%%%%%%%%%%%%%%%%%%%%%%%%%%%%%%%%%%%%%%%%%%%%%%%%%%%%%%%%%%%%%%%%%%%%%%%%%%%%%%%%%
%%%%%%%%%%%%%%%%%%%%%%%%%%%%%%%%%%%%%%%%%%%%%%%%%%%%%%%%%%%%%%%%%%%%%%%%%%%%%%%%%%%%%%%%%
%%
%%     S E C T I O N     5    S E C T I O N     5    S E C T I O N     5    S E C T I O N     5    S E C T I O N     5    S E C T I O N     5   
%%     S E C T I O N     5    S E C T I O N     5    S E C T I O N     5    S E C T I O N     5    S E C T I O N     5    S E C T I O N     5   
%%     S E C T I O N     5    S E C T I O N     5    S E C T I O N     5    S E C T I O N     5    S E C T I O N     5    S E C T I O N     5   
%%
%%%%%%%%%%%%%%%%%%%%%%%%%%%%%%%%%%%%%%%%%%%%%%%%%%%%%%%%%%%%%%%%%%%%%%%%%%%%%%%%%%%%%%%%%
%%%%%%%%%%%%%%%%%%%%%%%%%%%%%%%%%%%%%%%%%%%%%%%%%%%%%%%%%%%%%%%%%%%%%%%%%%%%%%%%%%%%%%%%%
\section{Conclusions}
In this paper we have reported on three redshift $z>2$ quasars with
dramatic changes in their \civ\ emission lines, the first
`Changing-Look'' quasars at high redshift.  This is also the first
time the changing-look behaviour has been seen in a high-ionization
emission line.

\begin{itemize}
\item SDSS J1205+3422, J1638+2827 and J2228+2201 show interesting behaviour
  in their observed optical light curves, and subsequent spectroscopy
  shows significant changes in the \civ\ broad emission line, with both
  line collapse and emergence being displayed in rest-frame timescales
  of $\sim$240-1640 days.
\item Where observed, the profile of the Ly$\alpha$/\nv emission complex
  also changes, and there is tentative evidence for changes in the \mgii
  line.
\item Although line measurements from the three quasars show large changes
  in the \civ\ line flux-line width plane, the quasars are not seen to
  be outliers when considered against the full $z>2$ quasar population
  in terms of (rest) Equivalent Width and FWHM properties.
\item 
  We put these observations in context with recent ``state-change''
  models, but note that even in their `low-state', the \civ\ CLQs are
  above $\sim10\%$ in Eddington luminosity.
\end{itemize}




\subsection*{Availability of Data and computer analysis codes} 
All materials, databases, data tables and code are fully available at: 
\href{https://github.com/d80b2t/CIV_CLQs}{\tt https://github.com/d80b2t/CIV\_CLQs}.


\section*{Acknowledgements}
NPR acknowledges support from the STFC and the Ernest Rutherford Fellowship scheme. 
%% KESF \& BM are supported by NSF PAARE AST-1153335. 
%% KESF \& BM thank CalTech/JPL for support during sabbatical.  
\\

\noindent
We thank:
\begin{list}{$\circ$}{}
  \item Andy Lawrence, Mike Hawkins and David Homan for useful discussion.
\end{list}

This paper heavily used \href{http://www.star.bris.ac.uk/~mbt/topcat/}{TOPCAT} (v4.4)
\citep[][]{Taylor2005, Taylor2011}.
%%
This research made use of \href{http://www.astropy.org}{\tt Astropy}, 
a community-developed core Python package for Astronomy 
\citep{AstropyCollaboration2013, AstropyCollaboration2018}.

Funding for SDSS-III has been provided by the Alfred P. Sloan
Foundation, the Participating Institutions, the National Science
Foundation, and the U.S. Department of Energy Office of Science. The
SDSS-III web site is
\href{http://www.sdss3.org/}{http://www.sdss3.org/}.
%%
SDSS-III is managed by the Astrophysical Research Consortium for the
Participating Institutions of the SDSS-III Collaboration including the
University of Arizona, the Brazilian Participation Group, Brookhaven
National Laboratory, Carnegie Mellon University, University of
Florida, the French Participation Group, the German Participation
Group, Harvard University, the Instituto de Astrofisica de Canarias,
the Michigan State/Notre Dame/JINA Participation Group, Johns Hopkins
University, Lawrence Berkeley National Laboratory, Max Planck
Institute for Astrophysics, Max Planck Institute for Extraterrestrial
Physics, New Mexico State University, New York University, Ohio State
University, Pennsylvania State University, University of Portsmouth,
Princeton University, the Spanish Participation Group, University of
Tokyo, University of Utah, Vanderbilt University, University of
Virginia, University of Washington, and Yale University.

This publication makes use of data products from the Wide-field
Infrared Survey Explorer, which is a joint project of the University
of California, Los Angeles, and the Jet Propulsion
Laboratory/California Institute of Technology, and NEOWISE, which is a
project of the Jet Propulsion Laboratory/California Institute of
Technology. WISE and NEOWISE are funded by the National Aeronautics
and Space Administration. No animals were harmed in the production of
this paper, but there was a large spider in NPRs apartment that
``vanished''.

\bibliographystyle{mnras}
\bibliography{tester_mnras}


% Don't change these lines
\bsp	% typesetting comment
\label{lastpage}
\end{document}


\end{document}