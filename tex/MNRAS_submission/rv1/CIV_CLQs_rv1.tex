\documentclass[fleqn,usenatbib]{mnras}

\input{format}

\title[High-redshift CLQs]{The first high-redshift changing-look quasars}

\author[Ross {\it et al.}]
{Nicholas~P.~Ross$^{1}$\thanks{E-mail: npross@roe.ac.uk},
Matthew J. Graham$^{2}$,
Giorgio Calderone$^{3}$, 
K. E. Saavik Ford$^{4,5,6}$,  
\newauthor Barry McKernan$^{4,5,6}$ and Daniel Stern$^{7}$  
\\
%%  List of institutions
$^{1}$Institute for Astronomy, University of Edinburgh, Royal Observatory, Blackford Hill, Edinburgh EH9 3HJ, United Kingdom \\
$^{2}$Cahill Center for Astronomy and Astrophysics, California Institute of Technology, Mail Code 249/17, 1200 E California Blvd, Pasadena CA 91125, USA\\
$^{3}$INAF -- Osservatorio Astronomico di Trieste, Via Tiepolo 11, I-34143 Trieste, Italy \\
$^{4}$Department of Science, BMCC, City University of New York, New York, NY 10007, USA \\
$^{5}$Department of Astrophysics, Rose Center for Earth and Space, American Museum of Natural History, Central Park West at 79th Street, NY 10024, USA \\
$^{6}$Graduate Center, City University of New York, 365 5th Avenue, New York, NY 10016, USA\\
$^{7}$Jet Propulsion Laboratory, California Institute of Technology, 4800 Oak Grove Drive, Mail Stop 169-221, Pasadena, CA 91109, USA \\
}

\date{Accepted XXX. Received YYY; in original form ZZZ}

\begin{document}
\label{firstpage}
\pagerange{\pageref{firstpage}--\pageref{lastpage}}
\maketitle

\begin{abstract}
We report on three redshift $z>2$ quasars with dramatic changes in
their \civ\ emission lines, the first sample of changing-look quasars
(CLQs) at high redshift.  This is also the first time the
changing-look behaviour has been seen in a high-ionisation emission
line.  SDSS J1205+3422, J1638+2827, and J2228+2201 show interesting
behaviour in their observed optical light curves, and subsequent
spectroscopy shows significant changes in the \civ\ broad emission
line, with both line collapse and emergence being displayed on
rest-frame timescales of $\sim$240-1640 days.  These are rapid
changes, especially when considering virial black hole mass estimates
of $M_{\rm BH} > 10^{9} M_{\odot}$ for all three quasars.  Continuum
and emission line measurements from the three quasars show changes in
the continuum-equivalent width plane with the CLQs seen to be on the
edge of the full population distribution, and showing indications of
an intrinsic Baldwin effect. We put these observations in context with
recent state-change models, and note that even in their observed
low-state, the \civ\ CLQs are generally above $\sim$5\% in Eddington
luminosity. 
\end{abstract}

\begin{keywords}
accretion, accretion discs -- surveys -- quasars: general -- quasars:
time-domain -- quasars: individual (SDSS~J120544.7+342252.4,
SDSS~J163852.9+28270.7.7, SDSS~J222818.7+220102.9)
\end{keywords}


%%%%%%%%%%%%%%%%%%%%%%%%%%%%%%%%%%%%%%%%%%%%%%%%%%%%%%%%%%%%%%%%%%%%%%%%%%%%%%%%%
%%%%%%%%%%%%%%%%%%%%%%%%%%%%%%%%%%%%%%%%%%%%%%%%%%%%%%%%%%%%%%%%%%%%%%%%%%%%%%%%%
%%
%%
%%   SECTION 1  SECTION 1  SECTION 1  SECTION 1  SECTION 1  SECTION 1  
%%   SECTION 1  SECTION 1  SECTION 1  SECTION 1  SECTION 1  SECTION 1  
%%   SECTION 1  SECTION 1  SECTION 1  SECTION 1  SECTION 1  SECTION 1  
%%
%%
%%%%%%%%%%%%%%%%%%%%%%%%%%%%%%%%%%%%%%%%%%%%%%%%%%%%%%%%%%%%%%%%%%%%%%%%%%%%%%%%%%
%%%%%%%%%%%%%%%%%%%%%%%%%%%%%%%%%%%%%%%%%%%%%%%%%%%%%%%%%%%%%%%%%%%%%%%%%%%%%%%%%%
\section{Introduction}
Luminous AGN, i.e. quasars, are now seen to significantly vary their
energy output on timescales as short as weeks to months.  This
observation, and the subsequent mismatch in the expected viscous
timescale, which for a $10^{7} M_{\odot}$ central supermassive black
hole (SMBH) is $\sim$hundreds of years, was noted over 30 years ago
\citep[e.g.][]{Alloin1985}. However, with new photometric light-curve
and repeat spectroscopic data, the desire for a deeper understanding
of AGN accretion disk physics has recently re-invigorated the field
\citep[e.g.,][]{Antonucci2018, Lawrence2018, Ross2018, Stern2018}.

The optical continuum variability of quasars was recognised since
their first optical identifications \citep{MatthewsSandage1963}. {\bf
However, dramatic changes} in the broad emission lines (BELs) of
quasars was only recently first identified \citep{LaMassa2015}.
Samples of over 100 ``changing-look'' quasars (CLQs) or
``changing-state'' quasars (CSQs) have now been assembled
\citep[e.g.,][]{MacLeod2019, Graham2020}. The community uses both
these terms as a cover for the underlying physics. For sake of
argument, CLQs can potentially be thought of as the extension to the
BELs of quasar continuum variability \citep[e.g., ][]{MacLeod2012}
whereas the CSQs have a state-transition, perhaps similar to that seen
in Galactic X-ray binaries \citep[][]{NodaDone2018, Ruan2019a}. In
this paper, we use the term ``changing-look'', as we are currently
agnostic, and confessedly ignorant, to the underlying physical
processes.

\begin{table*}
  \begin{centering}
    \begin{tabular}{l r  r    lll lll  r r r}
      \hline  \hline 
      Line                 & $\lambda$ &  Transition   &  \multicolumn{6}{c}{Transition Levels}                                                                                           & Wavenumber   & $A_{i,j}$                             & Physical             \\
                              &  / \AA\    & Energy / eV    &    \multicolumn{3}{l}{Lower; Conf., Term, J }    &  \multicolumn{3}{l}{Upper; Conf., Term, J}                                                         & / cm$^{-1}$    & / $10^{8}$ s$^{-1}$           &  Mechanism  \\      \hline
      H LyLim           &   912.324   & 13.5984       & 1$s$                    & $^2$S     & 1/2          & $\infty$                  &                                 &             & 109 678.7       & 1.23$\times10^{-6}$   & Ionisation \\
      H Ly$\alpha$  &  1215.670  & 10.1988       & 1$s$                     & $^2$S      & 1/2          & 2                             &                                 &             &  82 259.2       &  4.67  & Recombination \\
      \nv                  &  1238.821  & 10.0082       &  1$s^{2}$2$s$      &  $^{2}$S   &  1/2         & 1$s^{2}$2$p$          &  $^{2}$P$^{{\rm o}}$ &   3/2    &  80 721.9        & 3.40  & Collisional Ex. \\
      \nv                  &  1242.804  &  9.9762       &  1$s^{2}$2$s$       &  $^{2}$S   &  1/2        &  1$s^{2}$2$p$         &  $^{2}$P$^{{\rm o}}$ &  1/2     &  80 463.2        & 3.37 & Collisional Ex. \\
      \civ                 &  1548.187  &  8.0083       &  1$s^{2}$2$s$       &   $^{2}$S  & 1/2         & 1$s^{2}$2$p$          &  $^{2}$P$^{{\rm o}}$ &  3/2     &  64 591.7        & 2.65  &Collisional Ex. \\
      \civ                 &  1550.772  &  7.9950       & 1$s^{2}$2$s$        &  $^{2}$S   & 1/2         & 1$s^{2}$2$p$          &  $^{2}$P$^{{\rm o}}$ &  1/2     &   64 484.0       & 2.64  &Collisional Ex. \\
      \heii               &   1640.474  &  7.5578        & 2$p$ 	      &  $^{2}$P$^{{\rm o}}$ &  3/2  &  3$d$ 	                 & $^2$D                  &  5/2        &  60 958.0       & 10.35& Recombination \\
      \heii               &   1640.490  &  7.5578        & 2$p$ 	      &  $^{2}$P$^{{\rm o}}$ &  3/2  &  3$d$ 	                 & $^2$D                  &  3/2        &  60 957.4       & 1.73 &  Recombination \\
      \heii               &   1640.533  &  7.5576        & 2$p$ 	      &  $^{2}$P$^{{\rm o}}$ &  3/2  &  3$s$ 	                 & $^2$S                  &  1/2        &  60 957.4       & 0.68 &  Recombination \\
      \ciii                 &  1906.683  &  6.5026       & 1$s^{2}$2$s^{2}$   &   $^{1}$S   & 0            & 1$s^{2}$2$s$2$p$  &  $^{3}$P$^{{\rm o}}$ &   2        &   52 447.1       & 5.19$\times10^{-11}$ &Collisional  Ex. \\
      \ciii                 &  1908.734  &  6.4956       & 1$s^{2}$2$s^{2}$   &  $^{1}$S    & 0            & 1$s^{2}$2$s$2$p$  &  $^{3}$P$^{{\rm o}}$ &  1         &  52 390.8        & 1.14$\times10^{-6}$  &Collisional Ex. \\
       \mgii              &  2795.528  &  4.4338       & 2$p^{6}$3$s$        &  $^{2}$S    & 1/2        & 2$p^{6}$3$p$          &  $^{2}$P$^{{\rm o}}$ &   3/2     &  35 760.9       & 2.60  &Collisional Ex.\\
      \mgii               &  2802.705  &  4.4224       & 2$p^{6}$3$s$        &  $^{2}$S    & 1/2        & 2$p^{6}$3$p$          &  $^{2}$P$^{{\rm o}}$ &   1/2     &  35 669.3       & 2.57 & Collisional Ex.\\
      H Ba $\beta$   &  4861.333  &  2.5497       & 2                              &                 &               & 4                             &                                 &              &  20 564.8       & 0.0842 &  Recombination\\
      \hline   
      \hline
    \end{tabular}
    \caption{
      Strong UV/optical spectral emission lines in quasars, and their
      atomic data from the \href{https://physics.nist.gov/PhysRefData/ASD/lines_form.html}{NIST Atomic Spectra Database} 
      \citep{Kramida2018, Kramida2019}.  The transition energies are
      $E=hc/\lambda$ for the given wavelength. Transition level configurations are given in
      standard spectroscopic notation (as Configuration, Term and quantum J number) with $A_{i,j}$ denoting the transition
      probabilities.
      {\bf The physical mechanism for the generation of the emission lines
        e.g., full ionization, collisionally excitation (``Collisional Ex.'' in the rable) or recombination are also given in
      the last column.}
    }
    \label{tab:atomic_lines}
  \end{centering}
\end{table*}

\iffalse
\begin{table}
  \begin{centering}
    \begin{tabular}{l r}
      \hline
      \hline
      Transition & Energy Required \\
                      &   / eV (Ryd)\\
     \hline
     \ciii $\rightarrow $ \civ  & 47.89     (3.52) \\
      \civ   $\rightarrow $ \cv &  64.49   (4.74) \\
      \cv  $ \rightarrow $ \cvi & 392.09   (28.8) \\
      \niii $\rightarrow $ \niv  & 47.44  (3.49) \\
	\niv $\rightarrow $ \nv &  77.47  (5.69)  \\
      \nv $\rightarrow $ \nvi  &  97.89   (7.19) \\
            \hline
      \hline
    \end{tabular}
    \caption{Ionisation Energies for Carbon and Nitrogen.
	These are the energies that are needed to go 
	from the species on the left and create the ion on the right. }
  \end{centering}
\end{table}
\fi


CLQs to date have primarily been defined according to the
(recombination) Balmer emission line properties with particular
attention paid to the H$\beta$ emission line, observed from optical
spectroscopy. Recent works report on discoveries of \mgii\ changing-look
AGN \citep{Guo2019, Homan2020}. However, current CLQ studies have
primarily been at redshifts $z<1$.

While there have been many studies on triply ionised carbon, i.e.,
\civ, these have tended to focus on broad absorption line quasars
\citep[e.g., Table 1 of][]{Hemler2019} or the Baldwin effect
\citep{Baldwin1977, Bian2012, Jensen2016,
Hamann2017}. As noted in \citet{Rakic2017}, two different
types of Baldwin effect are present in the literature: {\it (i)} the {\it
global} (or {\it ensemble}) Baldwin effect, which is an
anti-correlation between the emission line equivalent width and the
underlying continuum luminosity of single-epoch observations of
a large number of AGN, and {\it (ii)} the {\it intrinsic} Baldwin
effect, the same anti-correlation but in an individual, variable 
AGN \citep{PoggePeterson1992}. The ensemble and intrinsic Baldwin effect 
has been observed in the \civ\ broad absorption line quasar population,
However, dramatic changes in the 
collisionally excited broad {\it emission} line of \civ\ --- and
indeed \ciii\ --- have not to this point been reported.

\begin{table*}
  \centering
  \begin{tabular}{l l   r ll  r r l}
    \hline 
    \hline 
    \multirow{2}{*}{Object} & \multirow{2}{*}{Redshift} & $g$-band & \multirow{2}{*}{MJD} & \multirow{3}{*}{Instrument} & Exposure      &   SDSS               & \multirow{2}{*}{Notes} \\
                                         &                                        &   / mag      &                                 &                                             &  time / sec &   Plate-FiberID  & \\
    \hline  
                                       & 2.068                              & 18.27          &  53498                   & SDSS                                     &  8057            & 2089-427             & Plate quality marginal \\
 J120544.7+342252.4     & 2.071                             &                    &  58538                   & DBSP                                     &  1800            &  ---                      &  Poor conditions \\
                                       & 2.071                             &                     &  58693                  & DBSP                                      &  2400            &  ---                      &  Good conditions        \\
                                       &                                       &                     &                               &                                              &                      &                               &                               \\
                                      & 2.185                              & 19.77           &  54553                   & SDSS                                     &   4801            &  2948-614              & Plate quality good \\
J163852.9+282707.7     & 2.186                              &                    &  55832                     & BOSS                                    &   3600            &  5201-178            & Plate quality good \\
                                      & 2.182                              &                     &  58583                     & LRIS                                    &  1800              &  ---                      & \\
                                      &                                         &                    &                                 &                                            &                      &                              &                                 \\
                                      & 2.217                               & 19.97          &  56189                    &  BOSS                                    &  2700            &   6118-720          & Plate quality good \\
 J222818.7+220102.9   & 2.222                               &                    &  56960                     & eBOSS                                    &  4500            &   7582-790          & Plate quality good \\ 
                                     &  2.222                               &                    &  58693                    &  DBSP                                    & 2400             &    ---                        &    \\
    \hline \hline   
  \end{tabular}
  \caption{Details of our spectroscopic observations.  Redshift errors are
    typically $\pm 0.002$.  SDSS, BOSS, and eBOSS spectra have
    resolving power $\mathcal{R} \equiv \lambda / \Delta \lambda \sim 2000$.  DBSP: Double Spectrograph on the Palomar
    200-inch Hale telescope.  LRIS: Low Resolution Imaging Spectrometer on Keck
    I 10-m telescope.} 
  \label{tab:obs_notes}
\end{table*}

Here, we present new results for three quasars which show dramatic
changes in their \civ\ and \ciii\ broad emission line properties and
underlying continuum. These are some of the first examples of CLQs at
high ($z>1$) redshift. Moreover, these are the first cases for
substantial changes of ions with high ionisation potentials (I.P.'s
$>$2 Rydberg), thus linking the ionizing photons to the energetic
inner accretion disk.

{\bf 
 Most of the strong lines in the rest-frame near-UV spectrum of quasars are
collisionally excited, including C\,{\sc iii}, C\,{\sc iv} and N\,{\sc
v}.  The \heii~$\lambda$1640 recombination line is among the exceptions.

Photoionization followed by recombination determines the ionization
level of carbon, and collisional excitation from the ions ground level 
determines the line intensity. Other processes do happen,
(including radiative and three body recombination to the upper levels 
of the transition), but they are much slower unless the density is
extremely high \citep[e.g., ][and H. Netzer,
priv. comm.]{Dopita_Sutherland2003book}. The ionization potential of C\,{\sc
iii} ion is 47.89 eV (3.519 Ry) and thus a photon with this energy
($k_{\rm B} T \sim 5.6 \times10^{5}$ K) can create the triply ionised,
\civ\ ion. Once the \civ\ ion is formed, collisional excitation leads to
the production of the 1548.202 and 1550.774~\AA\ \civ\ doublet.
}
%%
Details of the atomic transitions and physical mechanisms that produce
strong rest-frame UV/optical lines in quasars are given in
Table~\ref{tab:atomic_lines}.


\citet{Wilhite2006} examine \civ\ variability in a sample of 105
quasars observed at multiple epochs by the Sloan Digital Sky Survey
\citep[SDSS;][]{York2000, Stoughton2002, Abazajian2009}.  They find a
strong correlation between the change in the \civ\ line flux and the
change in the line width, but no correlations between the change in
flux and changes in line center or skewness.  These authors find that
the relation between line flux change and line width change is
consistent with a model in which a broad line base varies with greater
amplitude than the line core. The \civ\ lines in these high-luminosity
quasars appear to be less responsive to continuum variations than
those in lower luminosity AGN.

\citet{Richards2011} explored the broad emission line region in over
30,000 $z > 1.54$ SDSS quasars, concentrating on the properties of the
\civ\ emission line. These authors consider two well-known effects
involving the \civ\ emission line: {\it (i)} the anti-correlation
between the \civ\ equivalent width (EW) and luminosity (i.e., the global 
Baldwin effect) and {\it (ii)} the blueshifting of the peak of \civ\
emission with respect to the systemic redshift.  We denote the
velocity offset of emission lines as $V_{\rm off}$ and use the
convention that a positive $V_{\rm off}$ value means the line is
blueshifted while a negative $V_{\rm off}$ value means the line is
redshifted.  \citet{Richards2011} find a blueshift of the \civ\
emission line is nearly ubiquitous, with a mean shift of $\langle
V_{\rm off}\rangle \sim$810 km s$^{-1}$ for radio-quiet (RQ)
quasars and $\langle V_{\rm off}\rangle \sim$360 km s$^{-1}$
for radio-loud (RL) objects. \citet{Richards2011} also find the
Baldwin effect is present in both their RQ sample and their RL sample.
They conclude that these two \civ\ parameters (EW and blueshift) are
capturing an important trade-off between disk and wind components in
the disk-wind model of accretion disks \citep[e.g.,][]{Murray1995,
Elvis2000, Proga2000, Leighly2004b}, with one dominating over the
other depending on the shape of the quasar spectral energy
distribution (SED).

Using the multi-epoch spectra of 362 quasars from the SDSS
Reverberation Mapping project \citep[SDSS-RM; ][]{Shen2015, Shen2019},
\citet{Sun2018} investigate the blueshift of \civ\ emission relative
to \mgii\ emission, and its dependence on quasar properties. They
confirm that high-blueshift sources tend to have low \civ\ EWs, and
that the low-EW sources span a range of blueshift. Other
high-ionisation lines, such as \heii, also show similar blueshift
properties. The ratio of the line width of \civ\ to that of \mgii\
increases with blueshift. \citet{Sun2018} also find that quasar
variability might slightly enhance the connection between the \civ\
blueshift and EW, though further investigation here is warranted. They
also find that quasars with the largest blueshifts are less variable
and tend to have higher Eddington ratios, though Eddington ratio alone
might be an insufficient condition for the \civ\ blueshift. Recent
investigations also include \citet{Meyer2019} and
\citet{Doan2019}. \citet{Dyer2019} provide a detailed analysis of 340
quasars at high redshift ($1.62<z<3.30$) from the SDSS-RM project,
which we compare with our results in Section~\ref{sec:theory}.

%%    P a p e r    O v e r v i e w
The purpose of this paper is, for the first time, to systematically
access and report on the CLQ phenomenon at high ($z>2$)
redshift. While accessing this phenomenon at an earlier cosmic epoch
is interesting, the main value of this study is to move from
the low-ionisation energy Balmer emission line series at rest-frame
optical wavelengths to the high-ionisation emission lines, in
particular \civ\ $\lambda$1549, at rest-frame UV wavelengths.

This paper is organised as follows. In Section 2, we describe our
sample selection, catalogues, and observational data sets.  Section 3
presents the high-redshift quasars and reports their time-variable
line properties.  We provide a brief theoretical discussion in Section
4, and Section~5 presents our conclusions.  We report all magnitudes
on the AB zero-point system \citep{Oke_Gunn1983, Fukugita1996} unless
otherwise stated. For the {\it WISE} bands, $m_{\rm AB} = m_{\rm Vega}
+ m$ where $m = 2.699$ and 3.339 for {\it WISE} W1 (3.4$\mu$m) and W2
(4.6$\mu$m), respectively \citep{Cutri2011, Cutri2013}. We adopt a
flat $\Lambda$CDM cosmology with $\Omega_{\Lambda} = 0.73 $,
$\Omega_{\rm M} = 0.27$, and $h = 0.71$. All logarithms are to the
base 10.


%%%%%%%%%%%%%%%%%%%%%%%%%%%%%%%%%%%%%%%%%%%%%%%%%%%%%%%%%%%%%%%%%%%%%%%%%%%%%
%%%%%%%%%%%%%%%%%%%%%%%%%%%%%%%%%%%%%%%%%%%%%%%%%%%%%%%%%%%%%%%%%%%%%%%%%%%%%
%%
%%    SECTION 2   SECTION 2   SECTION 2   SECTION 2   SECTION 2   SECTION 2  
%%    SECTION 2   SECTION 2   SECTION 2   SECTION 2   SECTION 2   SECTION 2  
%%    SECTION 2   SECTION 2   SECTION 2   SECTION 2   SECTION 2   SECTION 2  
%%
%%%%%%%%%%%%%%%%%%%%%%%%%%%%%%%%%%%%%%%%%%%%%%%%%%%%%%%%%%%%%%%%%%%%%%%%%%%%%
%%%%%%%%%%%%%%%%%%%%%%%%%%%%%%%%%%%%%%%%%%%%%%%%%%%%%%%%%%%%%%%%%%%%%%%%%%%%%
\section{CLQ Selection and Line Measurements}
{\bf A statistical search using optical light curves to initiate spectroscopic follow-up
  observations is proving an efficient CLQ finder, and by design
our data sources and overall methodology, are in common to the recent
\citet{Ross2018}, \citet{Stern2018} and \citet{Graham2020} papers.
}
%
In this section we present the photometric data used to select the
CLQs and give details of the multiwavelength data where we have it.
{\bf We use data from the Catalina Real-time Transient Survey, PanSTARRS,
  the Zwicky Transient Facility, the WISE mission and SDSS.}
We then give details of the spectroscopic data including emission
lines measurements.

\subsection{Photometry}
\subsubsection{Optical Photometry}
We use optical data from the Catalina Real-time Transient Survey
\citep[CRTS;][]{Drake2009, Mahabal2011}, the Panoramic Survey
Telescope and Rapid Response System \citep[PanSTARRS;][]{Kaiser2010,
Stubbs2010, Tonry2012, Magnier2013}, and the Zwicky Transient Facility
\citep[ZTF;][]{Bellm2019_ZTFOverview}. 

The CRTS archive\footnote{http://catalinadata.org} contains the
Catalina Sky Survey data streams from three telescopes -- the 0.7-m
Catalina Sky Survey (CSS) Schmidt and 1.5-m Mount Lemmon Survey (MLS)
telescopes in Arizona, and the 0.5-m Siding Springs Survey (SSS)
Schmidt in Australia. CRTS covers up to $\sim$2500 deg$^2$ per night,
with four exposures per visit, separated by 10 min. The survey observes
over 21 nights per lunation. The data are broadly calibrated to
Johnson $V$ \citep[for details, see][]{Drake2013} and the current CRTS
data set contains time series for approximately 400 million sources to
$V \sim 20$ above Declination $\delta > -30$deg from 2003 to 2016 May (observed with CSS
and MLS) and 100 million sources to $V \sim 19$ in the Southern sky
from 2005 to 2013 (from SSS). CRTS has been extensively used to study 
quasar variability \citep[e.g.,][]{Graham2014, Graham2015, Graham2015Nature,
Graham2017, Graham2020, Stern2017, Stern2018, Ross2018}.

PanSTARRS data is obtained via the Pan-STARRS Catalog Search
interface\footnote{https://catalogs.mast.stsci.edu/panstarrs/}.
Specifically, we query the PS1 DR2 Detection catalog.

The Zwicky Transient Facility (ZTF; http://ztf.caltech.edu) is a new
robotic time-domain sky survey capable of visiting the entire visible
sky north of $-$30 deg declination every night. ZTF observes the sky in
the $g$, $r$, and $i$-bands at different cadences depending on the
scientific program and sky region \citep{Bellm2019_ZTFSurveys,
Graham2019_ZTF}. The ZTF 576 megapixel camera with a 47 deg$^{2}$
field of view, installed on the Samuel Oschin 48-inch Schmidt
telescope at Palomar observatory, can scan more than 3750 deg$^{2}$
per hour, to a 5$\sigma$ detection limit of 20.7 mag in the $r$-band
with a 30-sec exposure during new moon \citep{Masci2019}.
\begin{figure*}
  \centering
  %% trim=l b r t
  \includegraphics[width=16.7cm, trim=0.3cm 0.05cm 0.20cm 0.1cm, clip]
  {figures/J1205+3422_landscape_LC.png}
  \includegraphics[width=16.7cm, trim=0.3cm 0.05cm 0.20cm 0.1cm, clip]
  {figures/J1638+2827_landscape_LC.png}
  \includegraphics[width=16.7cm, trim=0.3cm 0.0cm  0.20cm 0.1cm, clip]
  {figures/J2228+2201_landscape_LC.png}
  \vspace{-6pt}
  \caption[]{The light curves for the three high-redshift CLQ quasars; 
    J1205+3422 (top), 
    J1638+2827 (middle) and  
    J2228+2201 (bottom). 
    The timing of the spectral observations are indicated with vertical dashed lines.}
  \label{fig:civ_clqs_LCs}
\end{figure*}

\subsubsection{Mid-Infrared Photometry}
Mid-infrared data (3.4 and 4.6$\mu$m) is available from the beginning
of the {\it WISE} mission \citep[2010 January; ][]{Wright2010} through
the fifth-year of {\it NEOWISE-R} operations \citep[2018 December;
][]{Mainzer2011}. The {\it WISE} scan pattern leads to coverage of the
full-sky approximately once every six months (a ``sky pass''), albeit
it with a gab between 2011 February and 2013 October when the
satellite was placed in hibernation. Hence, our light curves have a
cadence of 6 months with a 32-month sampling gap.

\begin{figure*}
  \centering
  %% trim=l b r t
  \includegraphics[width=16.7cm, trim=0.3cm 0.05cm 0.30cm 0.1cm, clip]
  {figures/J1205+3422_landscape_spectra.png}
  \vspace{-12pt}
  \caption[]{The three epochs of spectra for SDSS  J1205+3422.
    The full wavelength spectrum is presented in the top panel, 
    with zoom-in's on the Ly$\alpha$-\nv 
    complex, the \civ\ line, and the \mgii\ line in the lower panels}
  \label{fig:civ_clqs_J12_spec}
\end{figure*}

\begin{figure*}
  \centering
  %% trim=l b r t
  \includegraphics[width=16.7cm, trim=0.3cm 0.05cm 0.30cm 0.1cm, clip]
  {figures/J1638+2827_landscape_spectra.png}
  \vspace{-12pt}
  \caption[]{As Figure~\ref{fig:civ_clqs_J12_spec} but for  J1638+2827.}
  \label{fig:civ_clqs_J16_spec}
\end{figure*}

\begin{figure*}
  \centering
  %% trim=l b r t
  \includegraphics[width=16.7cm, trim=0.3cm 0.0cm  0.30cm 0.1cm, clip]
  {figures/J2228+2201_landscape_spectra.png}
  \vspace{-12pt}
  \caption[]{As Figure~\ref{fig:civ_clqs_J12_spec} but for   J2228+2201.}
  \label{fig:civ_clqs_J22_spec}
\end{figure*}

\subsection{CLQ Selection}
{\bf As this is the first time high-redshift CLQs have been studied,
defining a formal and automatic process is not an trivial task.  Hence
we focused on a method and threshold which selects the most promising
candidates, with no claim of completeness, in a number suitable to be
inspected visually.}

Our high-redshift CLQs were identified as follows.  We selected all
64,774 SDSS DR15 sources with $z > 0.35$ classified as `quasar', that
had at least two spectra separated by $\geq 100$ days, and that had a
corresponding CRTS light curve. We fitted a damped random walk to the
CRTS data via Gaussian process regression (using the
\href{https://gpy.readthedocs.io/en/deploy/}{GPy Python} library).
{\bf This allowed us to predict the photometric magnitudes and their
uncertainties at the epochs of the SDSS spectra for each source
\citep[see][]{Rasmussen_Williams2006}.}
%%
Candidate objects with $|\Delta V| > 0.3$ magnitude were then selected
for visual inspection, {\bf looking for at least a 33\% change in flux
between the two epochs. Approximately 1\% of quasars should show this
level of significant variability over the timescales considered
\citep{Graham2017}.}

Only three quasars, SDSS
J120544.7+342252.4 (hereafter J1205+3422), SDSS J163852.93+282707.7
(hereafter J1638+2827) and SDSS J222818.76+220102.9 (hereafter
J2228+2201), satisfied these selection criteria and showed interesting
or dramatic emission line behaviour.

%% From Section 3.1.
%% Placed here to help with overall formatting of the paper
\begin{table*}
  \small
  \begin{centering}
    \begin{tabular}{l  l | c c | c c c c | c}
      \hline
      \hline
                        &                   &  \multicolumn{2}{c}{Cont. @ 1450\AA }                                           &   \multicolumn{4}{c}{CIV 1549\AA}                                                                          &  Virial product \\
       Object      &   MJD           &       $\nu L_{\nu}$             &         Slope                                               &   Luminosity                &     FWHM               &    V$_{\rm off}$          &          EW        &  log($\nu L_{\nu})^{0.5} $ \\
                       &                     & $/10^{42}$ erg s$^{-1}$ & (F$_\lambda \propto \lambda^\alpha$)    & $/10^{42}$ erg s$^{-1}$ &    / km s$^{-1}$ &  /  km s$^{-1}$         &     /  \AA   & $\times$ FWHM$^{2}$   \\
      \hline
                       &  53498          &   41630   $\pm$   40    &  -1.70 $\pm$ 0.01                               &   700    $\pm$ 49       &  4700 $\pm$  120     &     580    $\pm$   80      &   27  $\pm$  1      &  9.66 $\pm$ 0.02\\
 J1205+3422  &  58538$^*$   &    2870    $\pm$   40     &  -1.29 $\pm$ 0.09                              &  170    $\pm$ 30      & 14900 $\pm$ 2800   &     980    $\pm$  690      &   92    $\pm$    16        & 10.08  $\pm$ 0.14 \\
                       &  58693          &    9050    $\pm$   10     &  -1.46 $\pm$ 0.00                              &  400   $\pm$   5       &  6980 $\pm$   90      &    1080    $\pm$   30      &   70 $\pm$  1  &  9.67 $\pm$ 0.01\\
                       &                      &                                      &                                                             &                                  &                                   &                                       &                                   &                  \\
                       &  54553         &    4060    $\pm$  120    &  -1.4  $\pm$ 0.2                                 & 330      $\pm$ 10      &  4630 $\pm$  190     &     180    $\pm$   60     &  128  $\pm$  4.4   &  9.14 $\pm$ 0.04\\
 J1638+2827   &  55832        &    2570    $\pm$   20      &  -2.16 $\pm$ 0.06                             & 100    $\pm$  5        &  4990 $\pm$  300      &     190    $\pm$   90      &   64  $\pm$  3.2   &  9.10 $\pm$ 0.05\\
                       &  58583        &    8600    $\pm$   20      &  -2.01 $\pm$ 0.01                             & 420    $\pm$  5        &  4620 $\pm$   70        &     100    $\pm$   20      &   81 $\pm$  0.93  &  9.30 $\pm$ 0.01\\
                       &                    &                                         &                                                          &                                   &                                   &                                        &                                    &                  \\
                       &  56189$^*$ &     860    $\pm$   40     &  -0.6  $\pm$ 0.1                                  &  40    $\pm$  10       &  5930 $\pm$  990      &     450    $\pm$  290      &   73  $\pm$  9.8   &  9.01  $\pm$ 0.13 \\
 J2228+2201  &  56960        &    9150    $\pm$   30     &  -1.88 $\pm$ 0.02                               &  340    $\pm$  10     &  7000 $\pm$  200     &    -255    $\pm$   60      &   60  $\pm$  1.4   &  9.67 $\pm$ 0.03\\
                      &  58693        &    2810  $\pm$    5        &  -1.38 $\pm$ 0.01                               &  160    $\pm$   5      &  5930 $\pm$   80       &     180    $\pm$   30      &   91 $\pm$  0.95  &  9.27 $\pm$ 0.01\\
      \hline
      \hline
      \hline
    \end{tabular}
    \caption{
      Continuum at 1450\AA\ and \civ\ spectral measurements for the
      three quasar considered in this work, at all observation epochs, as
      calculated by QSFit.  For the emission-line velocity offsets, a
      positive value means the line is blueshifted.  The last column shows
      the virial product calculated as $(\nu L_{\nu})^{0.5} \times {\rm
        FWHM}^2$.  {\bf The last column shows the virial product calculated as
        $(\nu L_{\nu})^{0.5} \times {\rm FWHM}^2$, and its uncertainty
        (calculated by propagating uncertainties on both $\nu L_{\nu}$ and the
        FWHM).  The virial products are calculated to check if the BH mass
        varies significantly across epoch.  } $^{*}$The \civ\ line is very
      faint (with respect to the continuum), and the associated estimates
      are likely unreliable.
    }
    \label{tab:QSFit-results}
  \end{centering}
\end{table*}

\subsection{Spectroscopy}
An overview of our spectroscopic observations is given in
Table~\ref{tab:obs_notes}.  The archival spectra are from the SDSS
\citep{Stoughton2002, DR7, Schneider2010}, the SDSS-III Baryon
Oscillation Spectroscopic Survey \citep[BOSS; ][]{Eisenstein2011,
Dawson2013, Smee2013, Alam2015, Paris2017} and the SDSS-IV Extended
Baryon Oscillation Spectroscopic Survey \citep[eBOSS; ][]{Dawson2016,
Abolfathi2018, Paris2018}.  These quasars were targetted via a range
of techniques and algorithms \citep[see][]{Richards2002, Ross2012,
Myers2015}. The SDSS, BOSS, and eBOSS data are supplemented by spectra
from the Low Resolution Imaging Spectrometer \citep[LRIS; ][]{Oke1995}
on the 10-m Keck {\sc I} telescope and the Double Spectrograph (DBSP)
instrument on the 200-inch Palomar Hale telescope.

Proper comparison of the spectra requires reliable flux calibrations,
which is a challenge since not all data were taken in photometric
conditions.  SDSS spectra are spectrophotometrically calibrated.  BOSS
and eBOSS spectra have spectrophotometric corrections applied in the
latest data release \citep{Hutchinson2016, Jensen2016, Margala2016}.
However, some of the Palomar and Keck data suffered from variable
conditions and clouds.  For lower redshift AGN, the narrow \oiii
emission line is typically used to align spectrophotometry since it is
spatially extended and not expected to change significantly on human
timescales \citep[e.g.,][]{Barth2011}. However, due to the
high-redshift of our quasars, \oiii is not available to us to use for
scaling the spectra.  Instead, we use photometric data from the ZTF
since our Palomar and Keck data were all taken after MJD 57500.  This
provides broad-band photometry at the time of the spectroscopy, which
we use to scale the spectra whose initial calibration is based on
spectrophotometric standards observed on the same nights.  {\bf We
note the resulting uncertainty in flux calibration is 2-5\%.  This
uncertainty is not propagated into measurements.}


\subsection{Emission Line and Power-law Slope Measurements}
We use the measured quasar emission line properties from several
catalogues: \citet{Shen2011}, \citet{Hamann2017},
\citet{Kozlowski2017}, and \citet{Calderone2017}.

In particular, we use the Quasar Spectral Fitting (QSFit) software
package presented in \citet{Calderone2017}. This provides luminosity
estimates as well as width, velocity offset and EWs of 20 emission
lines, including \civ, \ciii, and {Mg\,{\sc ii}}. We process and fit
all nine spectra (3 quasars with 3 epochs) using the lastest version
(v1.3.0) of the QSFit
\href{https://qsfit.inaf.it/cat_1.30/onlinefit.php}{online
calculator}. The host galaxy and blended iron emission at rest-frame
optical wavelengths components are automatically disabled when they
can not be constrained by the available data, as is the case for all
our quasars.  Power-law continuum slopes, $\alpha$, where $f_{\lambda}
\propto \lambda^{\alpha}$, are also reported in these catalogues and
from QSFit.

{\bf As described in \citet{Calderone2017}, the uncertainties in the
measured quantities are determined by two methods; a Fisher matrix
method and a Monte Carlo resampling method.  One might expect
correlations among model parameters, e.g. emission-line luminosities
and widths, and thus the Fisher matrix method to underestimates the
errors.  However, upon investigation, the parameters of the involved
model components (a power-law for the continuum, and a single
Lorentzian profile for the emission lines) show negligible
correlations, hence we use the Fisher matrix method to estimate the
uncertainties on the model parameters.  Section 2.8.3 and Appendix B4
of \citet{Calderone2017} gives further details and compares the Fisher
matrix and Monte Carlo resampling uncertainties and distributions.}
  



%%%%%%%%%%%%%%%%%%%%%%%%%%%%%%%%%%%%%%%%%%%%%%%%%%%%%%%%%%%%%%%%%%%%%%%%%%%%%
%%%%%%%%%%%%%%%%%%%%%%%%%%%%%%%%%%%%%%%%%%%%%%%%%%%%%%%%%%%%%%%%%%%%%%%%%%%%%
%%
%%   SECTION 3   SECTION 3   SECTION 3   SECTION 3   SECTION 3   SECTION 3  
%%   SECTION 3   SECTION 3   SECTION 3   SECTION 3   SECTION 3   SECTION 3  
%%   SECTION 3   SECTION 3   SECTION 3   SECTION 3   SECTION 3   SECTION 3  
%%
%%%%%%%%%%%%%%%%%%%%%%%%%%%%%%%%%%%%%%%%%%%%%%%%%%%%%%%%%%%%%%%%%%%%%%%%%%%%%
%%%%%%%%%%%%%%%%%%%%%%%%%%%%%%%%%%%%%%%%%%%%%%%%%%%%%%%%%%%%%%%%%%%%%%%%%%%%%
\section{Results}
\subsection{Photometric and Overall Spectral Evolution}
Figure~\ref{fig:civ_clqs_LCs} presents the optical and mid-infrared light
curves for three high-redshift CLQ quasars.

Figure~\ref{fig:civ_clqs_J12_spec} shows the rest-frame spectra for
SDSS J1205+3422.  For J1205+3422, our spectral observations cover 5195
observed days, corresponding to 1691 days in the rest-frame. This
quasar was initially identified in SDSS in 2005 May as a bright, $g
\approx 18.0$, blue-sloped quasar with broad Si\,{\sc iv}/O\,{\sc iv},
\civ, C\,{\sc iii}, and Mg\,{\sc ii}.  \civ\ has a blueshift of
$\approx$580$\pm$80 km s$^{-1}$.  By 2019, however, the optical
brightness dropped by nearly 2 magnitudes and the spectra are
significantly less blue.  \lya and \nv are detectable in both 2019
spectra, but is just blueward of the wavelength range covered by the
original SDSS spectrum. The \civ\ and \ciii\ lines have faded
significantly between the SDSS observation in 2005 and the Palomar
observations in 2019.  Note that \civ\ is extremely faint in the 2019
February spectrum, but that night suffered from poor conditions.

Figure~\ref{fig:civ_clqs_J16_spec} shows the rest-frame spectra for
SDSS J1638+2827.  For J1638+2827, our spectral observations cover 4030
observed days, corresponding to 1265 days in the rest-frame. Here, in
the initial epoch spectrum, \civ\ is broad and bright, as is C\,{\sc
iii}. However, $\approx$400 rest-frame days later, the broad \civ\ and
\ciii\ emission lines have faded, the continuum slope around 1400\AA\
has changed from $\approx-1.16$ to $\approx-2.17$, but the \lya/\nv
emission complex is very similar in shape {\bf and intensity}. Around
870 days in the rest-frame after the second spectral epoch, at the
time of the third spectral epoch, the source has brightened from
optical magnitudes of $\sim$20 mag to $\sim$18.5 mag. Ly$\alpha$,
N\,{\sc v}, \civ, \ciii, and \mgii\ are all apparent and broad, with
\mgii\ being seen for the first time at high signal-to-noise. An
absorption feature between \lya and \nv is seen in all three spectral
epochs.

Figure~\ref{fig:civ_clqs_J22_spec} shows the rest-frame spectra for
SDSS J2228+2201.  For J2228+2201, our spectral observations cover 2504
observed days, corresponding to 778 days in the rest-frame. Over the
course of 240 rest-frame days, \civ\ and \ciii\ both {\it emerge} as
BELs and {\bf the standard UV/blue continuum becomes stronger.}  Then,
over the course of 538 days in the rest-frame, the broad emission,
while still very present, reduces in line flux the UV/blue continuum
diminishes.  However, the third epoch still corresponds to a source
more luminous than the initial BOSS spectrum.

\begin{figure*}
  \centering
  \includegraphics[width=18.40cm, trim=0.37cm 0.3cm 0.0cm 0.2cm, clip]{figures/CIVregions.png}
    \vspace{-18pt}
  \caption{Observed spectra and best fit models of the \civ\ spectral region.
    The solid blue line is the Lorentzian profile fit to the \civ\ emission line.
    The long-dashed red line is the continuum fit, with the solid orange
    line giving the overall fit. The short-dashed vertical lines gives the
    rest wavelengths of the \civ\ doublet. }
  \label{fig:QSFit-CIV}
\end{figure*}

\subsection{\civ\ Emission Line Evolution}
We analyse the multi-epoch spectra of the three quasars using the
QSFit spectral fitting package \citep{Calderone2017}.  One advantage
of using QSFit is that it allows constraints on the slope and
luminosity of the broad band continuum of the source. The relevant
estimated quantities, including continuum luminosity and slope at
rest-frame 1450\AA, \civ\ line luminosity, FWHM, and EW are given in
Table~\ref{tab:QSFit-results}.  {\bf All fits are obtained by taking
the Milky Way extinction \citep{Schlafly2011} into account.  The
extinction values are: $E(B-V) =0.013$ for J1205+3422; $E(B-V) =0.034$
for J1638+2827 and $E(B-V) =0.041$ for J2228+2201, as taken from the
\href{http://ned.ipac.caltech.edu/}{NASA/IPAC Extragalactic Database} 
}. The best fit models in the region of the \civ\ emission line are
shown in Figure~\ref{fig:QSFit-CIV}.

All \civ\ lines are fitted with a single, broad, Lorentzian profile.
This allows us to account for the central peak of the \civ\ line.  No
additional narrow components are considered for several reasons.
First, in the epochs of highest brightness, such a ``narrow''
component would have FWHM $\sim$ 2000-3000~km~s$^{-1}$, i.e. values
exceeding the usual widths of genuine narrow lines ($\lesssim
1000$~km~s$^{-1}$). Second, allowing a second component to have such
large widths would make that component highly degenerate with the
``broad'' components, causing the latter to have much larger widths,
$\sim$10,000~km~s$^{-1}$.  Finally, neglecting such a narrow
component ensures a consistent model across all epochs.

Both the quasar continuum (evaluated at 1450\AA), and the \civ\ line
luminosities follow a similar evolution, with a ratio of $\sim$20-30,
confirming that the main driver for emission line variability is
likely the broad band continuum itself.  For all sources except
J1638+2827, the slope of the continuum changes with luminosity
following a ``bluer-when-brighter'' pattern, suggesting that a
distinct emerging component is responsible for both the slope and
luminosity variations.  In J1638+2827 the opposite behaviour is
observed, especially in the first observation epoch.  However, this
may be a bias due to the limited wavelength range available which
extends to rest-frame $\lambda \sim 1240$~\AA\ for the first epoch,
while it extend to shorter wavelengths for the other epochs
(respectively ~\AA\ and 1010~\AA).  This suggests that the emerging
component is more prominent at UV wavelengths, and a sufficient
wavelength coverage is required to detect it. In all cases where the
\civ\ line profile is reliably constrained, we find that the \civ\
FWHM is approximately constant with maximum variations $\lesssim 1000$
km s$^{-1}$, despite significantly larger line luminosity variations.

\begin{figure}
  \centering
  \includegraphics[width=8.5cm, trim=0.2cm 0.2cm 0.0cm 0.2cm, clip]
  {figures/CIV_CLQs_MBHvsREW_20200519.png}
   \vspace{-12pt}
   \caption[]{
     Colored density cloud shows virial black hole masses of
     $\approx$20,000 $z>1.5$ quasars from QSFit catalogue
     \citep{Calderone2017} compared to their \civ\ rest-frame equivalent
     widths (REWs).  Overplotted are the three high-redshift CLQs
     considered here: J1205+3422 (magenta), J1638+2827 (green), and
     J2228+2201 (cyan).  The symbol shapes indicate the observation epoch,
     with the first, second, and third epochs denoted by circles, squares,
     and diamonds, respectively. As per Table~\ref{tab:QSFit-results}, an
     asterisk in the inset box signifies that the \civ\ line for that
     quasar observation is very weak, likely making the associated
     estimates unreliable.}
   \label{fig:CIV_MBHvsREW}
\end{figure}

\subsection{Virial Black Hole Masses}\label{sec:BH_masses} 
The FWHMs of quasar broad lines are related to the mass of the SMBH
powering the quasar phenomenon, and that mass is assumed to be
constant on any human timescale.  Hence it is instructive to check
whether the virial products, which are the basic quantity used to
calculate single-epoch black hole mass estimates, show any
variation. 

This approach assumes that the broad-line region (BLR) is virialized
\citep[e.g.][]{Shen2008b, Shen2011, Calderone2017,
Mejia-Restrepo2018}.  {\bf However, the \civ\ line is observed to have
line asymmetry due to outflows \citep[e.g.,][]{Gaskell1982}, very
strong blueshifts \citep[as mentioned in the Introduction, see
][]{Richards2011}, and a component of the line profile that does not
reverberate \citep[][]{Denney2012}.  These issues make the virial BH
mass estimate uncertain. \citet{Runnoe2013}, 
\citet{Mejia-Restrepo2016}, \citet{Coatman2017}, 
\citet{Mejia-Restrepo2018} and \citet{Grier2019} all present studies
into improving \civ-based single-epoch black hole mass estimates. With
these cavaets in mind, we acknowledge the BH mass estimates will have
a large error, likely $\sim$0.5 dex. }
The continuum
luminosity provides a proxy for the BLR radius, and the broad-line
width (FWHM) provides a proxy for the virial velocity.  This ``virial
mass'' estimate is then expressed as:
{\bf
\begin{equation}
%  \log(M) = \log \left[c\left(  \frac{\nu L_\nu}{10^{42} {\rm erg\ s^{-1}}} \right)^{\gamma}  \left( \frac{\rm FWHM}{10^3 {\rm km\ s^{-1}}} \right)^2 \right] + {\rm constant}.
  \log \left( \frac{M_{\rm BH}}{M_{\odot}} \right)  = a + \log  \left[ \; \left(  \frac{L_{1450}} {10^{44} \; {\rm erg \; s^{-1}}}\right)^{\gamma} \, \right] + 2 \log \left( \frac{\rm {FWHM}}{{\rm km\; s^{-1}}} \right)
  \label{eqn:virial_masses}
\end{equation}
\citet{VestergaardPeterson2006} have $\gamma=0.53$ and the offset as $a=0.660$, 
which is widely adopted \citep[e.g.,][]{Chen2019, Yao2019}.
\citet{Kozlowski2017} have similar values \citet{VestergaardPeterson2006} , with $(a,\gamma)=(0.64, 0.53)$,
but note there is a problem with the FWHMs for \civ\ for those quasars that are reported in
both the DR7 Quasar catalog and DR12 Quasar catalog.
\citet{Grier2019} measure a slope of $\gamma=0.51\pm0.05$ for their sample of high-redshift
$z>1.3$ quasars that show significant lag measurements,
and  $\gamma=0.52\pm0.04$ for their ``gold sample'' of 16 of
their highest-confidence objects. }

We report a virial product for the \civ\ CLQs in the last column of
Table~\ref{tab:QSFit-results}.  {\bf Since our purpose is to check if
this measurement varies significantly across epochs, here we use
$\gamma=0.5$ as the power-law slope and are not directly concerned
with the offset.  The quoted uncertainties account only for the
propagation of uncertainties in the continuum luminosities and line
widths.  As noted above, with typical uncertainties associated to
single epoch virial mass around $\sim$0.5 dex,} this implies that the
virial products for our three high-redshift CLQs are all remarkably
constant and compatible with single black hole masses per source, even
in those cases where the \civ\ estimates were deemed potentially
unreliable. The object showing the largest variation is J2228+2201,
with the BH mass estimate spanning a range of 0.66~dex.

{\bf In Table~\ref{tab:Eddington_ratios} we report the \civ\ Virial
masses calculated using equation~\ref{eqn:virial_masses}, the measured
luminosites and FWHM from QSFit and the
\citet{VestergaardPeterson2006} calibrations.} \citet{Shen2011} and
\citet{Kozlowski2017} report virial black hole masses based on \mgii\
and \civ\ for the SDSS/BOSS observations of our targets. For
J1205+3422 (and MJD 53498), \citet{Shen2011} report $\log (M_{\rm BH, MgII}/M_{\odot})
= 9.55 \pm 0.05$ and $\log(M_{\rm BH, CIV}/M_{\odot}) = 9.49 \pm
0.04$.



We use the mean of five values --- three from our three epochs
and two from \citet{Shen2011} --- to calculate an adopted SMBH mass of
$\log (M_{\rm BH, vir}/M_{\odot}) = 9.73$ for J1205+3422.  J1638+2827
also has two virial mass measurements from \citet{Shen2011}, as well
as two measurements from \citet{Kozlowski2017}. The mean mass
measurement adopted for J1638+2827 is $\log(M_{\rm BH, vir}/M_{\odot}
= 9.09$. For J2228+2201, there are two \citet{Kozlowski2017}
measurements, leading to a mean adopted mass of $\log(M_{\rm BH,
vir}/M_{\odot} = 9.37$. We use these mean SMBH masses when calculating
the Eddington ratios.

From the virial mass estimates, all our objects have SMBH masses
$M_{\rm BH} >10^{9} M_{\odot}$.  This is at the upper end of SMBH
masses at all epochs, and towards the extreme of the mass distribution
for $z\sim2$ quasars.  Figure~\ref{fig:CIV_MBHvsREW} compares the
\civ\ EW versus the virial SMBH masses for our multi-epoch
observations to a sample of $\approx$20,000 $z>1.5$ SDSS quasars from
the QSFit catalog \citep{Calderone2017}.

{\bf The values in Table~\ref{tab:QSFit-results} are
$\approx$0.27-0.30 dex higher than the estimates in
Table~\label{tab:Eddington_ratios}, which is to be expected given the
difference of calibrations factors. In fact, the virial products are
all remarkably constant and compatible with a single black hole mass
estimate, within the standard 0.5 dex uncertainty, across the 3 epochs
for all the quasars.  This fact gives us confidence in the line
fitting methods.}



\begin{figure}
  \centering
  \includegraphics[width=8.5cm, trim=0.2cm 0.2cm 0.0cm 0.2cm, clip]
  {figures/CIV_CLQs_REWvsFWHM_20200519.png}
  \vspace{-12pt}
  \caption[]{Rest-frame EW vs. FWHM of \civ\
    for BOSS DR12 quasar sample \citep{Hamann2017}.
   Symbols as in Figure~\ref{fig:CIV_MBHvsREW}.}
  \label{fig:REWvsFWHM}
\end{figure}

\subsection{Quantified Temporal Evolution of \civ\ Emission}
Quasars with interesting physical properties, such as extreme
outflows, can be selected using EW and FWHM measurements \citep[e.g.,
``Extremely Red Quasars'' ---][]{Ross2015, Zakamska2016, Zakamska2019,
Hamann2017}. Figure~\ref{fig:REWvsFWHM} shows the rest-frame EW versus
the FWHM of the \civ\ emission line in the BOSS DR12 quasar sample
from the catalogue of \citet{Hamann2017}.  Other than the two low
quality, suspect observations, the multi-epoch observations of our
high-redshift CLQs are all consistent with the bulk of the BOSS DR12
sample.

The temporal evolution of the velocity offsets, 1450~\AA\ continuum
luminosity, continuum slope, \civ\ line width (FWHM), and virial black
hole mass estimates are shown in Figure~\ref{fig:QSFit-results}.  The
\civ\ velocity offsets are approximately constant, and generally
compatible with a single value (within 3$\sigma$).  The exception is
J2228+2201, where a significant change ($\sim$7$\sigma$) is observed
between the second and third epochs. The velocity blueshifts ($\sim
200-1100$ km s$^{-1}$) are consistent with rest-frame UV spectra of
quasars over the redshift range $1.5 \leq z \leq 7.5$
\citep[e.g.,][]{Meyer2019}.


\begin{figure*}
  \centering
  \includegraphics[width=\textwidth]{figures/QSFit-results}
  \vspace{-12pt}
  \caption{
    Temporal evolution of the spectral properties of the three quasars
    considered in this work (see Table 3).  The third row show the
    logarithmic virial product with two error bars: the propagated
    uncertainties for the $(\nu L_{\nu})^{0.5} \times {\rm FWHM}^2$
    estimates (red) and the typical uncertainty associated to single epoch
    virial masses (0.5 dex, black).}
  \label{fig:QSFit-results}
\end{figure*}


\begin{table}
  \centering
  \begin{tabular}{l l  cc    l     r}
    \hline
    \hline
\multirow{2}{*}{Object} & \multirow{2}{*}{MJD} & \multicolumn{2}{c}{$\log (M/M_{\odot})$} & log($L_{\rm bol}$) &  \multirow{2}{*}{$\eta_{\rm Edd}$}  \\
                                    &                                  &    QSFit               &   Liter. value               &                             &    \\
    \hline
                                    & 53498                        &    9.39                &  9.49$\pm$0.04$^{\dagger}$   &   47.22                  &  $-$0.61   \\
 J1205+3422               & 58538                         &    9.78               &   ---                                        &  46.07                  &   $-$1.76 \\  %% L_Edd (J12)  = 47.82637   for log(M) = 9.726
                                    & 58693                        &    9.38               &   ---                                               &  46.57                &  $-$1.26  \\                    
    \hline 
                                    & 54553                        &   8.84                &  8.74$\pm$0.15$\dagger$                     & 46.17   & $-$1.03  \\
J1638+2827                 & 55832                       &    8.80                &    9.13$^{\ddagger}$                                         & 45.98   & $-$1.21  \\      %% L_Edd (J16)  = 47.191370  for log(M) = 9.091
                                    & 58583                        &   9.01                &  ---                         & 46.50   & $-$0.69   \\
    \hline 
                                     & 56189                       & 8.70                 & 8.73$^{\ddagger}$                         & 46.23  &  $-$1.24 \\
J2228+2201                 & 56960                       &  9.39                &   ---                       & 47.34  & $-$0.12  \\      %% L_Edd (J22)  = 47.466370  or log(M) = 9.366
                                     & 58693                      &  8.97                    &    ---                      & 46.83   &  $-$0.64 \\
    \hline
    \hline
  \end{tabular}
  \caption{Virial Mass estimates, bolometric luminosities and Eddington ratios for the
    three quasars. 
    $\dagger$From the \citet{Shen2011} SDSS DR7 Quasar catalog.
    $\ddagger$From the \citet{Kozlowski2017} SDSS-III BOSS DR12 quasar catalog.
%    are given in either \citet{Shen2011} or    \citet{Kozlowski2017} for the three CLQs.
    We scale these values using
    our measured $\nu L_{\nu}$ continuum luminosity values.  From
    Section~\ref{sec:BH_masses}, we assume: $\log(M_{\rm BH,
      vir}/M_{\odot}) = 9.726$ for J1205+3422; $\log(M_{\rm BH,
      vir}/M_{\odot}) = 9.091$ for J1638+2827, and $\log(M_{\rm BH,
      vir}/M_{\odot}) = 9.366$ for J2228+2201.  The Eddington luminosity
    $L_{\rm Edd} = 1.26\times10^{38} (M_{\rm BH}/M_{\odot})$ erg s$^{-1}$
    and $\eta_{\rm Edd}$ is the log of the Eddington ratio.}
\label{tab:Eddington_ratios} 
\end{table}


\begin{figure}
  \centering
  \includegraphics[width=8.5cm, trim=0.2cm 0.2cm 0.0cm 0.2cm, clip]{figures/CIV_CLQs_Baldwin_20200519.png}
   \vspace{-12pt}
   \caption[]{\civ\ equivalent width versus underlying continuum luminosity,
     commonly referred to as the Baldwin plot.  The continuum luminosities
     are from \citet{Calderone2017}.  The rest-frame EW (REW) measurements
     are from Table~\ref{tab:QSFit-results}.  Symbols as in
     Figure~\ref{fig:CIV_MBHvsREW}.
     % The dashed red line has slope     $\beta=-0.38$.
   }
  \label{fig:CIV_Baldwin}
\end{figure}

\subsection{The \civ\ Baldwin Effect}
The Baldwin effect \citep{Baldwin1977} is an empirical relation
between quasar emission line rest-frame EWs and continuum luminosity
\citep[e.g.,][]{Shields2007, Hamann2017, Calderone2017}.
\citet{Hamann2017} and \citet{Calderone2017} present recent
measurements of the Baldwin effect for large quasar samples.

There is an anti-correlation between the rest-frame EWs and, e.g.,
1450~\AA\ rest-frame continuum luminosity, so that as the underlying
UV continuum luminosity increases, the EW decreases.
Figure~\ref{fig:CIV_Baldwin} shows this for a sample of 20,374 quasars
from the QSFit catalogue \citep{Calderone2017}.  The slope is $\beta=
-0.20$, consistent with \citet[][ $\beta=-0.25$, but for bolometric
luminosity rather than rest-frame UV continuum luminosity]
{Kozlowski2017} and with \citet[][ $\beta=-0.23$]{Hamann2017}.

We add the measurements from the three \civ\ CLQ at each epoch
to Figure~\ref{fig:CIV_Baldwin}. All three quasars at all three epochs
lie on the edge of the $\nu L_{\nu}$-EW distribution.  Also, with the
exception of J1638+2827 on MJD 55832, all the measurements show an
{\it intrinsic} Baldwin effect \citep[e.g.,][]{Goad2004, Rakic2017}.
%% Subject to small number statistics, the slope of the intrinsic Baldwin effect for these three high-redshift \civ\ CLQs is somewhat steeper than the population trends, with $\beta \approx-0.38$, shown by the dashed red line in Figure~\ref{fig:CIV_Baldwin}.



%%%%%%%%%%%%%%%%%%%%%%%%%%%%%%%%%%%%%%%%%%%%%%%%%%%%%%%%%%%%%%%%%%%%%%%%%%%%%
%%%%%%%%%%%%%%%%%%%%%%%%%%%%%%%%%%%%%%%%%%%%%%%%%%%%%%%%%%%%%%%%%%%%%%%%%%%%%
%%
%%   SECTION 4   SECTION 4   SECTION 4   SECTION 4   SECTION 4   SECTION 4  
%%   SECTION 4   SECTION 4   SECTION 4   SECTION 4   SECTION 4   SECTION 4  
%%   SECTION 4   SECTION 4   SECTION 4   SECTION 4   SECTION 4   SECTION 4  
%%
%%%%%%%%%%%%%%%%%%%%%%%%%%%%%%%%%%%%%%%%%%%%%%%%%%%%%%%%%%%%%%%%%%%%%%%%%%%%%
%%%%%%%%%%%%%%%%%%%%%%%%%%%%%%%%%%%%%%%%%%%%%%%%%%%%%%%%%%%%%%%%%%%%%%%%%%%%%
\section{Discussion}
\subsection{Continuum and Line Changes: Comparisons to recent Observations}
The top row of Figure~\ref{fig:QSFit-results} demonstrates that both
the 1450~\AA\ continuum and the \civ\ emission lines can exhibit
large, greater than an order-of-magnitude, changes in luminosity, {\it
and} that these continuum-line changes track each other.

\citet{Trakhtenbrot2019} report on the quasar 1ES~1927+654 at
$z=0.02$, which was initially seen to lack broad emission lines and
line-of-sight obscuration, i.e, it was a ``type 2'' quasar. This
object was then seen to spectroscopically change with the appearance
of a blue, featureless continuum, followed by the emergence of broad
Balmer emission lines --- i.e., this quasar changed into a broad-line,
or ``type 1'' quasar, after the continuum luminosity brightened.  This
implies that there is (at least in some cases) a direct relationship
between the continuum and broad emission lines in CLQs.  Many other
examples of a similar phenomenon have been reported, albeit not with
the dense spectral cadencing of 1ES~1927+654.  A similar scenario may
have occurred for the three high-redshift CLQs presented here,
although we lack the high-cadence, multiwavelength, multi-epoch
coverage that \citet{Trakhtenbrot2019} present.  Interestingly,
\citet{Trakhtenbrot2019} find no evidence for broad UV emission lines
in their source, including neither \civ, \ciii, nor \mgii.  The
authors attribute the lack of broad UV emission lines to dust within
the BLR.

\citet{MacLeod2019} present a sample of CLQs where the primary
selection requires large amplitude ($| \Delta g | > 1$ mag, $| \Delta
r | > 0.5$ mag) variability over any of the available time baselines
probed by the SDSS and Pan-STARRS1 surveys. They find 17 new CLQs,
corresponding to $\sim 20$\% of their observed sample. This CLQ
fraction increases from 10\% to roughly half, as the rest-frame
3420~\AA\ continuum flux ratio between repeat spectra increases from
1.5 to 6 --- i.e., more variable quasars are more likely to exhibit
large changes in their broad emission lines. \citet{MacLeod2019} note
that their CLQs have lower Eddington ratios relative to the overall
quasar population.

Using the same dataset as \citet{MacLeod2019}, \citet{Homan2020}
investigate the responsiveness of the \mgii\ broad emission line
doublet in AGN on timescales of several years.  Again focussing on
quasars that show large changes in their optical light-curves,
\citet{Homan2020} find that \mgii\ clearly does respond to the
continuum.  However, a key finding from \citet{Homan2020} is that the
degree of responsivity varies strikingly from one object to another.
There are cases of \mgii\ changing by as much as the continuum, more
than the continuum, or very little at all.  In the majority (72\%) of
this highly variable sample, the behaviour of \mgii\ corresponds with
that of H$\beta$.  However, there are also examples of \mgii\ showing
variations while but H$\beta$ does not, and vice versa.

\citet{Graham2020} report the largest number of H$\beta$ CLQs to
date, with a sample of 111 sources identified. \citet{Graham2020}
find that this population of extreme varying quasars, referred to as
CSQs in their paper, is associated with {\it changes} in the Eddington
ratio rather than simply the magnitude of the Eddington ratio itself.
They also find that the timescales imply cooling/heating fronts
propagating through the disk.


\subsection{Continuum and Line Changes: Comparisons to Theoretical Expectations}\label{sec:theory}
The \civ\ line is one of the strongest collisionally excited lines in
quasar spectra \citep[e.g.,][]{HamannFerland1999}.
{\bf Early measurements from e.g., 
  \citet{PoggePeterson1992} and \citet{Peterson1997book}, 
  using the well-studied Seyfert galaxy NGC 5548,
  suggested the location of \civ\ emission was 
%  probes the photoionisation environment
  produced by the innermost disk, with the line emission
  responding to the underlying continuum emission in only
  a matter of $\sim$ 8 days. 
as indicated by reverberation mapping time-delay measurements.
%{\bf (a) I would recommend citing some representative papers to support the statement made in the first sentence of section 4.2 about the lag of the C IV line. The most appropriate papers are those dealing with quasars of comparable luminosity (and probably comparable redshift) as the quasars studied here. I am somewhat skeptical of this statement because the papers that I consulted.}
However, more recent measurements from e.g. \citet[][and references therein]{Grier2019},
suggest that the lag between the \civ\ line and the continuum are of order $>100$ observed-frame days. 
which would place the \civ emitting region far from the inner accretion disk.
Importantly, many quasars in the \citet{Grier2019} sample are of
comparable redshift and luminosity to SDSS J1205+3422, J1638+2827, and J2228+2201.}


In standard \citet{SS73} thin disk models, large changes in the
continuum flux are not permitted over short timescales due to the
relatively long viscous time associated with such disks. Given the
observed short timescale continuum variations, we must consider more
complex models. We note that two of our sources (J1205+3422 and
J2228+2201) have \civ\ EW and continuum luminosity changes which fall
comfortably along a line, implying their variation is consistent with
an intrinsic Baldwin effect (see Figure~\ref{fig:CIV_Baldwin}).
%%
{\bf The light-crossing time of the region that produces the \civ\ line is substantial,
which translates into a delay between the variations of the continuum and
the \civ\ line, and a smoothing of the light curve of the latter.
%%
Thus, if the excitation of the \civ\ line is driven by the changes in the
continuum, the excited gas cannot reach an equilibrium
excitation state as quickly as the continuum changes.}
This makes the slope of the intrinsic Baldwin effect greater than that
of the overall ensemble slope, which is derived from many single-epoch
observations, the majority of which are assumed to be in equilibrium.
{\bf \citet{Grier2019}, }


The presence of an intrinsic Baldwin effect implies those sources may
comfortably fit into the sample of \civ\ variable quasars explored by
\citet{Dyer2019}. Similar to those authors, we consider slim accretion
disk models \citep[e.g.,][]{Abramowicz1988}, which may explain the
observed variability. In particular, the summary of disk variation
timescales presented in \citet{Stern2018} shows that timescales are
shorter for taller disks, permitting changes similar to those observed
if they are caused by heating or cooling fronts propagating on the
disk sound crossing time. \citet{Dyer2019} also consider inhomogeneous
disk models \citep[e.g.,][]{DexterAgol2011} where flux variations are
driven by azimuthal inhomogeneities in the temperature of the
disk. Such inhomogeneities could arise from disk instabilities
\citep[e.g.,][]{LightmanEardley1974} or interactions of embedded
objects \citep[e.g.,][]{McKernan2014, McKernan2018}. If an
inhomogeneous disk is responsible for the continuum variations, which
then drive \civ\ variations, it implies that our objects are simply
the extreme outliers produced by a process which occasionally, but
rarely, produces very large hot or cool spots. The frequency of such
occurrences can be used to constrain the slope and normalization of
the power law distribution of spot size.

Finally, we must also consider the disk/wind model which has
successfully reproduced many features of quasar \civ\ observations,
notably the common blueshifted offset \citep[e.g.,][]{Murray1995}. In
this model, \civ\ is optically thick at low velocities, but optically
thin at high velocities, with $\tau \sim 1$ at $\sim$5000 km
s$^{-1}$. In a disk/wind model, the BLR is very small, implying short
timescales of emission line variability. Also in this model, there
should be associated strong absorption in the soft X-ray band, which
could be tested with follow-up observations.

The foregoing discussion directly applies to the two objects which
maintain an intrinsic Baldwin effect relationship between \civ\ EW and
continuum luminosity. However, J1638+2827 clearly does not maintain
such a relationship --- indeed, the continuum and \civ\ EW appear
somewhat correlated, rather than anti-correlated, in this source. This
implies that the illumination of the \civ\ emitting region by a
variable ionising continuum and the corresponding change in the
photoionisation state alone cannot explain the collapse and recovery
of the line.


\begin{figure*}
  \centering
  \includegraphics[width=14.5cm, trim=0.2cm 0.2cm 0.0cm 0.2cm, clip]
  {figures/MJD_vs_Eddington_20191211}
  \vspace{-12pt}
  \caption[]{Eddington ratios of the three \civ\ CLQs.  Grayscale gives the
    Eddington ratio ranges for quasars from SDSS \citet[][]{Shen2011} and
    BOSS \citep[][]{Kozlowski2017}.  Symbols as in
    Figure~\ref{fig:CIV_MBHvsREW}.  The red region ($L / L_{\rm Edd}
    \approx 0.02 \pm 0.01$; $\eta_{\rm Edd} = -1.7 \pm 0.13$) is the
    transition accretion rate suggested by \citet{NodaDone2018}.  A
    ``High/Soft State'' with $\eta_{\rm Edd} \geq -1$ and a a ``Low/Hard
    State'' with $\eta_{\rm Edd} \leq -2$ are indicated with dashed lines
    and arrows.}
  \label{fig:Eddington_ratios}
\end{figure*}

\subsection{Eddington Ratios and State Changes} 
The broad UV and optical lines in quasars are most sensitive to the
extreme ultraviolet (EUV) part of the SED, with \civ\ (and indeed 
\heii\ and N\,{\sc v}) being at the higher energy end of the EUV
distribution.

\citet{NodaDone2018} note that the soft X-ray excess -- the excess of
X-rays below 2 keV with respect to the extrapolation of the hard X-ray
spectral continuum model, and a common feature among Type~1 AGN --
produces most of the ionizing photons, so its dramatic drop can lead
to the disappearance of the broad-line region, driving the
changing-look phenomena. \citet{NodaDone2018} go on to make a
connection between state changes in Galactic binaries and CLQs.
%
In Galactic binaries spectral hardening
corresponds to the soft-to-hard state transition in black hole
binaries, which occurs at $L / L_{\rm Edd} \sim 0.02$ (i.e.,
$\eta_{\rm Edd} \sim -1.7$).  During this state transition, the inner
disc evaporates into an advection-dominated accretion flow, while the
overall drop in luminosity appears consistent with the hydrogen
ionisation disc instability.  \citet{NodaDone2018} predict that
changing-look AGNs are similarly associated with state transitions at
$L / L_{\rm Edd}$ of a few percent.

Comparing observed correlations between the optical-to-X-ray spectral
index (i.e., $\alpha_{\rm ox}$) and Eddington ratio in AGN to those
predicted from observations of X-ray binary outbursts,
\citet{Ruan2019a} find a remarkable similarity to accretion state
transitions in prototypical X-ray binary outbursts, including an
inversion of this correlation at a critical Eddington ratio of
$\sim$10$^{-2}$, i.e., at the same ratio motivated by
\citet{NodaDone2018}.  These results suggest that the structures of
black hole accretion flows directly scale across a factor of
$\sim10^{8}$ in black hole mass and across different accretion
states. Using \citet{Ruan2019a} as a guide, there are potentially
three accretion regimes: {\it (i)} a ``High/Soft State'' with
$\eta_{\rm Edd} \gtrsim -1$; {\it (ii)} a ``Low/Hard State'' with $-2
\lesssim \eta_{\rm Edd} \lesssim -1$; and {\it (iii)} a ``Low/Hard
State'' with $\eta_{\rm Edd} \lesssim -2$.  These are given as shaded
regions in Figure~\ref{fig:Eddington_ratios}.

Apart from J1205+3422 on MJD 58538, we see all the \civ\ CLQ epochs
are above $L / L_{\rm Edd} \sim 5$\%. As we have noted throughout, the
spectrum for J1205+3422 on MJD 58538 has low SNR, so while this object
could well have entered a `low-state', this is difficult to
conclusively confirm.  Nevertheless, and taking the J1205+3422
spectrum at face-value, we note that all the \civ\ CLQs are well above
$\sim1\%$ in Eddington luminosity and thus the suggested ``Low/Hard
State'' boundary.

While we note there is interest in plotting these accretion rates, and
motivation from \citet{NodaDone2018} and \citet{Ruan2019a}, we very
much caution on over-interpretation at this juncture. The point of
this paper was to report on three very interesting CLQs.  Our object
definitions are based on empirical, observed properties.  The
potential for physical connections of accretion physics across the
large-range of mass-scales is tantalising, but is left to future
investigations.



%%%%%%%%%%%%%%%%%%%%%%%%%%%%%%%%%%%%%%%%%%%%%%%%%%%%%%%%%%%%%%%%%%%%%%%%%%%%%%%%%%%%%%%%%
%%%%%%%%%%%%%%%%%%%%%%%%%%%%%%%%%%%%%%%%%%%%%%%%%%%%%%%%%%%%%%%%%%%%%%%%%%%%%%%%%%%%%%%%%
%%
%%     S E C T I O N     5    S E C T I O N     5    S E C T I O N     5    S E C T I O N     5    S E C T I O N     5    S E C T I O N     5   
%%     S E C T I O N     5    S E C T I O N     5    S E C T I O N     5    S E C T I O N     5    S E C T I O N     5    S E C T I O N     5   
%%     S E C T I O N     5    S E C T I O N     5    S E C T I O N     5    S E C T I O N     5    S E C T I O N     5    S E C T I O N     5   
%%
%%%%%%%%%%%%%%%%%%%%%%%%%%%%%%%%%%%%%%%%%%%%%%%%%%%%%%%%%%%%%%%%%%%%%%%%%%%%%%%%%%%%%%%%%
%%%%%%%%%%%%%%%%%%%%%%%%%%%%%%%%%%%%%%%%%%%%%%%%%%%%%%%%%%%%%%%%%%%%%%%%%%%%%%%%%%%%%%%%%
\section{Conclusions}
In this paper we have reported on three redshift $z>2$ quasars with
dramatic changes in their \civ\ emission lines, the first sample of
changing-look quasars at high redshift.  This is also the first time
the changing-look behaviour has been seen in a high-ionisation
emission line.

\begin{itemize}
\item SDSS J1205+3422, J1638+2827 and J2228+2201 show interesting
  behaviour in their observed optical light curves, and subsequent
  spectroscopy shows significant changes in the \civ\ broad emission
  line, with both line collapse and emergence being displayed on
  rest-frame timescales of $\sim 240-1640$~days.
\item Where observed, the profile of the Ly$\alpha$/\nv emission complex
  also changes, and there is tentative evidence for changes in the \mgii\
  line.
\item Although line measurements from the three quasars show large changes
  in the \civ\ line flux-line width plane, the quasars are not seen to
  be outliers when considered against the full quasar population
  in terms of (rest-frame) EW and FWHM properties.
\item 
  We put these observations in context with recent ``state-change''
  models, and note that with the exception of the low SNR spectrum of
  J1205+3422 on MJD 58538, even in their low state, 
  the \civ\ CLQs are above $\sim5$\% in Eddington luminosity. 
\end{itemize}

There are now more observed examples of dramatic, dynamic changes in
supermassive black hole systems e.g. CLQs/CSQs than Galactic X-ray
Binaries.  While the timescales are expected to scale with black hole
mass, whether the full physical system at large does, including the
associated atomic physics, remains an open question.


\subsection*{Availability of Data and computer analysis codes} 
All materials, databases, data tables and code are fully available at: 
\href{https://github.com/d80b2t/CIV_CLQs}{\tt https://github.com/d80b2t/CIV\_CLQs}.


\section*{Acknowledgements}
We thank: \\
\begin{itemize}
\item Andy Lawrence, Mike Hawkins and David Homan for useful discussion;
\item the referee for a very constructive report that improved the paper;
\end{itemize}
NPR acknowledges support from the STFC and the Ernest Rutherford Fellowship scheme.
% KESF \& BM are supported by NSF PAARE AST-1153335. 
% KESF \& BM thank Caltech/JPL for support during sabbatical.
MJG is supported in part by the NSF grants AST-1815034, and the NASA grant 16-ADAP16-0232.

This paper heavily used \href{http://www.star.bris.ac.uk/~mbt/topcat/}{TOPCAT} (v4.4)
\citep[][]{Taylor2005, Taylor2011}.
%%
This research made use of \href{http://www.astropy.org}{\tt Astropy}, 
a community-developed core Python package for Astronomy 
\citep{AstropyCollaboration2013, AstropyCollaboration2018}.
\href{https://pylustrator.readthedocs.io/en/latest/}{\tt pylustrator} \citep{Gerum2019} was also used. 

Funding for SDSS-III has been provided by the Alfred P. Sloan
Foundation, the Participating Institutions, the National Science
Foundation, and the U.S. Department of Energy Office of Science. The
SDSS-III web site is
\href{http://www.sdss3.org/}{http://www.sdss3.org/}.
%%
SDSS-III is managed by the Astrophysical Research Consortium for the
Participating Institutions of the SDSS-III Collaboration including the
University of Arizona, the Brazilian Participation Group, Brookhaven
National Laboratory, Carnegie Mellon University, University of
Florida, the French Participation Group, the German Participation
Group, Harvard University, the Instituto de Astrofisica de Canarias,
the Michigan State/Notre Dame/JINA Participation Group, Johns Hopkins
University, Lawrence Berkeley National Laboratory, Max Planck
Institute for Astrophysics, Max Planck Institute for Extraterrestrial
Physics, New Mexico State University, New York University, Ohio State
University, Pennsylvania State University, University of Portsmouth,
Princeton University, the Spanish Participation Group, University of
Tokyo, University of Utah, Vanderbilt University, University of
Virginia, University of Washington, and Yale University.

This publication makes use of data products from the {\it Wide-field
Infrared Survey Explorer}, which is a joint project of the University
of California, Los Angeles, and the Jet Propulsion
Laboratory/California Institute of Technology, and {\it NEOWISE}, which is a
project of the Jet Propulsion Laboratory/California Institute of
Technology. {\it WISE} and {\it NEOWISE} are funded by the National Aeronautics
and Space Administration.

Based on observations obtained with the Samuel Oschin 48-inch Telescope at the Palomar Observatory as part of the Zwicky Transient Facility project. ZTF is supported by the National Science Foundation under Grant No. AST-1440341 and a collaboration including Caltech, IPAC, the Weizmann Institute for Science, the Oskar Klein Center at Stockholm University, the University of Maryland, the University of Washington, Deutsches Elektronen-Synchrotron and Humboldt University, Los Alamos National Laboratories, the TANGO Consortium of Taiwan, the University of Wisconsin at Milwaukee, and Lawrence Berkeley National Laboratories. Operations are conducted by COO, IPAC, and UW.
%% No animals were harmed in the production of this paper, but there
%% was a large spider in NPRs apartment that ``vanished''.
 
%NPR dedicates this work, and indeed his previous research to his father. Dad had a deep love for science, mathematics and questioning the conventional wisdom. His support allowed me to become an astrophysicist and to have the adventures and discoveries we did. He is sorely missed. 


\bibliographystyle{mnras}
\bibliography{tester_mnras}

% Don't change these lines
\bsp	% typesetting comment
\label{lastpage}
\end{document}


\end{document}