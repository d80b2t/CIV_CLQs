\documentclass[11pt,a4paper]{article}

\input{../../tex/format}

\newcommand{\qsfit}{{\sc QSFit}}

\begin{document}

\noindent
So, for br\_CIV\_1549 (NCOMP=1) I start with something like::
  \begin{center}
    \begin{tabular}{l   r    r   r  r  r  r  r  r    }
      \hline
                   & & & & & & &    \\
   LUM        & LUM\_ERR &  FWHM   & FWHM\_ERR   &   VOFF  & VOFF\_ERR  &    EW         &  EW\_ERR  \\
   510.57   &  13.16       & 9592.9  &  287.34         &   1171.7 &  73.785      &     15.898  & 0.40976    \\
                   & & & & & & &    \\
      \hline
\end{tabular}
\end{center}
from the \qsfit{} output log, e.g. from {\tt
spec-2089-53498-0427\_QSFIT\_z2p068.log}.


\begin{equation}
  \mu  = 1548 \; \AA
\end{equation}
Assuming $c$ has units of km s$^{-1}$, $\mu$ is the nominal working wavelength (1548 \AA\ for \civ\ )
\begin{equation}
    {\rm FWHM}_{\lambda}  = ( 9592.9  /3 \times 10^{5}) * \mu 
  \end{equation}
  $\Rightarrow$
\begin{equation}
  {\sigma}     =   {\rm FWHM}_{\lambda}  / 2. (\sqrt(2 * \ln(2)))
\end{equation}
and I now just have $\mu$ and $\sigma$ to plug into the Gaussian PDF: 
\begin{equation}
  f(x\mid \mu ,\sigma ^{2})={\frac {1}{\sqrt {2\pi \sigma ^{2}}}}e^{-{\frac {(x-\mu )^{2}}{2\sigma ^{2}}}}
\end{equation}
and where $x$ is in Angstroms.
I'm assuming $x$ will be $x' = x - V_{\rm Off}$ at some point, but I'm not there yet(!).
% \begin{lstlisting}
 % mu = 1549.
 % ##  $c$ in km/s; nominal wavelength in Ang
%linefit_x =  np.linspace(mu - 3*qsfit_sigma[ii], mu + 3*qsfit_sigma[ii], 100)
%linefit_y  = stats.norm.pdf(linefit_x, mu, qsfit_sigma[ii])
%\end{lstlisting}

\vspace{16pt}
\noindent
Now, for the line luminosites, I have: 
\begin{eqnarray}
     f &  = &   L / (4 \; \pi   \;  D_{L}^{2})  \\
  &  = &   (5.1057 \times 10^{44} [{\rm erg  \;  s^{-1}}] ) /   (4 \; \pi \; ((5.094 \times 10^{28})^{2} [{\rm cm}])) \\
         & = & 1.56 \times 10^{-14}  [{\rm erg  \;  s^{-1} cm^{-2}} ]
\end{eqnarray}
where $L$ is (line) luminosity, $D_{L}$ is Luminosity Distance
($z=2.071$), $f$ is flux.  So I think I'm still missing 'something'
here as the e.g. SDSS data are in $10^{-17} {\rm erg \; s^{-1}
cm^{-2}}$, where the peak \civ\ luminosity for J1205+3422 (MJD 53498)
is around 45 in these units.

\vspace{34pt}
\noindent
Python code (pseudo or actual!)::  
\begin{lstlisting}
import matplotlib.pyplot as plt
import numpy as np
import scipy.stats as stats
import math

mu        = 0
variance = 1
sigma      = math.sqrt(variance)
x         = np.linspace(mu - 3*sigma, mu + 3*sigma, 100)
plt.plot(x, stats.norm.pdf(x, mu, sigma))
plt.show()
  \end{lstlisting}  


\end{document}